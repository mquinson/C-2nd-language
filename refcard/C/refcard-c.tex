\documentclass[a4paper,10pt,landscape]{article}
\RequirePackage{verbatim}
\RequirePackage{ifthen}
\RequirePackage{fancyhdr}
\RequirePackage{fancyvrb}

\RequirePackage{figlatex}
\RequirePackage{graphics}
\usepackage[utf8]{inputenc}

\usepackage{multicol,url,enumitem,hyperref}
\hypersetup{hidelinks}
\usepackage[margin=1cm]{geometry}
\setlength{\parindent}{0pt}
\setlength\columnseprule{0.5pt}

\newcommand{\Section}[1]{{\Large \textbf{#1}}}
\newcommand{\Subsection}[1]{{\textbf{#1}}}


\begin{document}\pagestyle{empty}
\centerline{\Large\bf Mémento du langage C}
\medskip
\begin{minipage}{1.0\linewidth}
\begin{multicols}{3}
  \Subsection{Compiler} (= produire un exécutable)%\vspace{-.4\baselineskip}
  \begin{itemize}[noitemsep,nolistsep]
  \item Tout en un: \hfill\texttt{gcc -Wall -Wextra -g aze.c -o aze}
  \item (1) compilation: \hfill\texttt{gcc -Wall -Wextra -g -c aze.c}
  \item (2) édition de liens: \hfill\texttt{gcc aze.o toto.o -o prog}
  \end{itemize}

  \medskip\Subsection{Pré-processeur} (= cherche/remplace automatique)\\
  \verb=  /* bloc de commentaires */=   \\
  \verb=  // Commentaire jusqu'à la fin de ligne= \\
  \verb=  #include <libmodule.h>=\\
  \verb=  #include "usermodule.h"=\\
  \verb=  #define TAILLE 512=\\
  \verb=  #define MAX(a,b) ((a)>(b)?(a):(b))=

  \medskip\Subsection{Blocs d'instructions} (= morceaux de programme)
  \vspace{-.3\baselineskip}
  \begin{Verbatim}[gobble=2,commandchars=+\[\]]
    [+textit[expression];          // instruction]
    { +textit[instructions]+ldots }  // bloc
    if (+textit[expression]) {+textit[bloc]}
    if (+textit[expression]) {+textit[bloc]} else {+textit[bloc]}
    switch (+textit[expression]) {
      case +textit[constante1]: +textit[instructions]+ldots break;
      case +textit[constante2]: +textit[instructions]+ldots break;
      +textit[default]: +textit[instructions]+ldots
    }
    while  (+textit[expression]) {+textit[bloc]}
    for (+textit[init] ; +textit[condition]; +textit[increment]) {+textit[bloc]}
    do  {+textit[bloc]} while  (+textit[cond]) // rarement utile
    break; // termine le bloc (boucle ou switch)
    continue; // prochaine itération
    return +textit[expr]; // la fonction retourne la valeur
  \end{Verbatim}

  \medskip\Subsection{Structure des programmes} (= contenu des fichiers)
  \begin{itemize}[noitemsep,nolistsep]
  \item Fichiers d'entête: que des déclarations\\
    \textcopyright, \verb=#include=, \verb=#define=, prototypes fctions globales
  \item Fichiers d'implém: déclarations et implémentations\\
    \textcopyright, \verb=#include=, \verb=#define=, prototypes fctions
    statiques, définition des fonctions
  \item Fichier principal: comme implem + fonction main\\
    \verb=int main(int argc, char **argv)=    
  \end{itemize}

  \medskip\Subsection{Identificateurs} (= noms de variables et fonctions)
  \begin{itemize}[noitemsep,nolistsep]
  \item Lettre suivie par des lettres ou chiffres ou souligné\\
    \verb=[a-zA-Z][a-zA-Z0-9_]*=
  \item Mots réservés~: {\small auto$\,$ break$\,$ case$\,$ char$\,$
      const$\,$ continue default do double else entry enum extern float
      for goto if int long register return short signed sizeof static
      struct switch typedef union unsigned void volatile while}
  \end{itemize}

  %%%%%%%%%%%%%%%%%%%%%%%%%%%%%%%%%%%%%%%%%%%%%%%%%%%%%%%%%%%%%%%%%%%%%%%
  \columnbreak
  \Subsection{Déclarations, initialisations et prototypes} 
  \vspace{-.6\baselineskip}
  \begin{Verbatim}[gobble=2]
    int i = 0;             char c = '\n';
    char* str = "hello";   char buf[BUFSIZ];
    double x = 3.14;       char* p = NULL;
    double values[MAX] = {1, 2, 3};
    typedef enum { FALSE, TRUE } Bool;
    typedef struct KeyVal {
      char* key;
      unsigned int val;
    } KeyValType;
    KeyValType klist[ ] = {{"NSW", 0}, {"Vic", 5}};
    void compute_ranking(int from);
  \end{Verbatim}  
  
  \smallskip\Subsection{Exemples de littéraux} (= valeurs de variables)
  \vspace{-.5\baselineskip}
  \begin{tabbing}
    ~~~\=\verb=123  0xAf0C  057  5L  3ul = \= entiers (\verb=int=)\\
    \>\verb=3.1415   3f  1.29e-23=   \> réels (\verb=double=)\\
    \>\verb='x'   '\t'   '\033'=     \> caractères (\verb=char=)\\
    \>\verb="hello"  "abc\"\n"  ""=  \> chaînes (\verb=char *=)
  \end{tabbing}

  \smallskip\Subsection{Séquences d'échappement} (dans les chaînes)
  \vspace{-.5\baselineskip}
  \begin{tabbing}
    ~~~~\=\verb=\t  = \= tabulation~~~~~~~~~~~~~ \=\verb=\n  = \= nouvelle ligne\\
    \>\verb=\'=  \> caractère \verb='=  \>
      \verb=\"=  \> caractère \verb="=  \\
    \>\verb=\\=  \> caractère \verb=\=  \>
      \verb=\123=  \> code ascii en octal  \\
    \>\verb=\0= \> caractère nul \>liste complète: \verb=man ascii= 
  \end{tabbing}
  
  \Subsection{Taille des types} (en fonction de l'architecture)

  ~~~\begin{minipage}{.65\linewidth}
    \begin{tabular}{|l |c c c|}\cline{2-4}
      \multicolumn{1}{c|}{}&x86&win64& amd64\\\hline
      \verb=short=     & 2 &  2  & 2\\\hline
      \verb=int=       & 4 &  4  & 4\\\hline
      \verb=long=      & 4 &  4  & 8\\\hline
      \verb=long long= & 8 &  8  & 8\\\hline
      pointeurs        & 4 &  8  & 8\\\hline
    \end{tabular}
  \end{minipage}
  \begin{minipage}{.3\linewidth}
    
    (en octets)

    cf. \verb=sizeof()=\par
    \verb=  char:= tjs 1\par
    \verb= float:= tjs 4\par
    \verb=double:= tjs 8
  \end{minipage}

  \medskip\Subsection{Portée et durée de vie des identificateurs}
  \smallskip

  ~~~\begin{tabular}{|l |c|c|}\cline{2-3}
    \multicolumn{1}{c|}{} & Portée  & Durée de vie\\\hline
    Globale               & partout & infinie \\\hline
    Globale \verb=static= & fichier seulement & infinie \\\hline
    Locale \verb=static=  & bloc seulement & infinie \\\hline
    Locale                & bloc seulement & bloc \\\hline
  \end{tabular}
  
  ~~~Les globales sont \verb=extern= par défaut.\\[-.5\baselineskip]

  \textbf{Bibliothèque standard:} \verb=#include <stdlib.h>= \vspace{-.6\baselineskip}
  \begin{tabbing}
    ~~~\=\verb=atoi(s)  = \=\verb=atof(s)    =\=Chaîne vers \verb=int= ou \verb=double=\\
    \>\verb=abs(n)  = val absolue\>\>\verb=rand()  =nb pseudo-aléatoire\\
    \>\verb=malloc(n)= \>\verb=calloc(1,n)= \>~Alloue $n$ octets\\
    \>\verb=realloc(p,n)=\>\>Redimensionne $p$\\
    \>\verb=free(p) = Libère bloc 
    \>\>\verb=exit(n)= Termine prog (code $n$)
  \end{tabbing}
  \columnbreak %%%%%%%%%%%%%%%%%%%%%%%%%%%%%%%%%%%%%%%%%%%%%%%%%%%%%%%%

  \Subsection{Opérateurs} (priorité décroissante)
  
  \begin{tabular}{|c|c| p{53.5mm}|}\hline
    \multicolumn{2}{|c|}{\tt ()  [ ]  .  -> }
    & parenthèses, tableau, structure, \hfill$\rightarrow$\par
      structure pointée\\\hline
    \multicolumn{2}{|c|}{++ ~ -- ~~  - ~~ !}&Incr/décrément, moins unaire, non\\
    \multicolumn{2}{|c|}{* ~~ \& ~~ \~}&déréférencemnt, adresse,
                                         complem.1$\!\!$\\
    \multicolumn{2}{|c|}{sizeof() ~~  (\textit{type})}&
     taille d'objet, transtypage \hfill$\leftarrow$\\\hline
    * / \% & + - & Opérateurs arithmétiques\hfill$\rightarrow$ \\\hline
    \multicolumn{2}{|c|}{$<<$ ~~~ $>>$} &Décalage binaire à gauche/droite\hfill$\rightarrow$\\\hline
    \multicolumn{2}{|c|}{$<$ ~ $<=$ ~ $>=$ ~ $>$} &Opérateurs relationnels\hfill$\rightarrow$\\\hline
    ==  != &  \& & Tests d'(in)égalité; ET binaire\hfill$\rightarrow$\\\hline
    \verb=^= & \verb=|= & OU exclusif (XOR); OU inclusif\hfill$\rightarrow$\\\hline
    \multicolumn{2}{|c|}{\textit{C} ? \textit{V} : \textit{F}}&
        Condition ternaire\hfill$\leftarrow$\\\hline
    \multicolumn{2}{|c|}{=  += -= *= \&= }&
        Affectation (avec modification)\hfill$\leftarrow$\\\hline
    \multicolumn{2}{|c|}{,}& Virgule (séquence d'expressions)\hfill$\rightarrow$\\\hline
  \end{tabular}

  \vspace{-\baselineskip}  
  \begin{tabbing}
    $c$~ \=\verb=char = \= $p$~ \=pointeur~ \=~$s$ \=chaîne \verb=char*=~~ \=$d$ \=\verb=double=\\
    $n$ \> \verb=int= \>$l$\>\verb=long= \>$\!\!$$fh$ \>fichier \verb=FILE*= \>$b$\>buffer \verb=char[]=
  \end{tabbing}
  \vspace{-\baselineskip}

  \textbf{I/O standard}: \verb=#include <stdio.h>=\vspace{-.7\baselineskip}
  \begin{tabbing}
    ~\=\verb=stdin= \=\verb=stdout= \=\verb=stderr   =   \=Flux de sortie (\verb=FILE*=)\\
    \>\verb=EOF= \>\verb=NULL= \>\>Des constantes utiles\\
    \>\verb=fopen(s, "rwab+")= \>\>\>Ouvre fichier, retourne un $fh$\\
    \>\verb=fclose(fh)=\>\>\>Ferme le fichier\\
    \pushtabs
    ~\=\verb=fgetc(fh)   getchar() =\=Lit un cara (\verb=EOF= si fini)\\
    \>\verb=fputc(fh,c) putchar(c)=\>Écrit un caractère\\
    \>\verb=fread(b,1,n,fh)=\>Lit un bloc de taille au plus $n$\\
    \>\verb=fwrite(b,1,n,fh)=\>Écrit un bloc de taille $n$\\
    \>\verb=printf(fmt, list)=\>Affichage formaté sur $stdout$\\
    \>\verb=fprintf(fh,fmt, list)=\>Affichage formaté sur $fh$\\
    \>\verb=sprintf(b,fmt, list)=\>Affichage formaté dans buffer\\
    \>\verb=scanf(fmt, list)=\>Lecture formaté depuis $stdin$\\
    \>\verb=fscanf(fh, fmt, list)=\>Lecture formaté depuis $fh$\\
    \>\verb=sscanf(s, fmt, list)=\>Lecture formaté depuis chaîne
    \poptabs
  \end{tabbing}
  \vspace{-.8\baselineskip}
  ~Format de printf: \%\textit{[largeur][}.\textit{precision]}\textit{type}
  \vspace{-.8\baselineskip}
  \begin{tabbing}
    ~~~~~\=$d$~~ \=entier~~~~~\=$o$~~\=octal~~~~~~ \=$x$~~\=hexadécimal\\
    \>$f$\>flotant\>$g$\>général\>$e$\>exponentiel (scientifique)\\
    \>$c$\>caractère\>$s$\>chaîne\>$p$\>pointeur
  \end{tabbing}
  \vspace{-.8\baselineskip}
  ~Pour \verb=scanf=, la liste contient des adresses

  \medskip
  \textbf{Utiliser des chaînes:} \verb=#include <string.h>= \vspace{-.6\baselineskip}
  \begin{tabbing}
    ~~\=\verb=strlen(s)   = \=longueur (sans le \verb=\0=)\\
    \>\verb=strcmp(s1,s2)=\>~~compare. ~~1:$<_{lex}$  ~~~0:$==_{lex}$ ~~~-1:$>_{lex}$\\
    \>\verb=strcpy(d,s)=\>copie d$\leftarrow$s \verb=  strcat(d,s)= ajout d$\leftarrow$ds\\
    \>\verb=strchr(s,c)=\>cherche $c$ \verb=   strstr(s,s2)= cherche $s2$\\
    \>\verb=strtok(s,delim) =Découpe $s$ en tokens
  \end{tabbing}
\end{multicols}  
\end{minipage}%
\begin{minipage}{.1\linewidth}  
  \rotatebox{90}{
    \scriptsize Licence: CC-BY-SA, by M. Quinson. Voir
    \url{https://github.com/mquinson/C-2nd-language}

     ~~~~~v20170214
  }
\end{minipage}

\setlength\columnseprule{0pt}

\begin{multicols}{3}
  \VerbatimInput[label=Premier programme, frame=single,
  numbers=left,numbersep=2pt]{premier_programme.c}

  \smallskip
  \begin{Verbatim}[label=Lecture d'un fichier ligne à ligne, gobble=4, frame=single,
    numbers=left,numbersep=2pt]
    char* ligne = NULL;
    size_t lgr = 0;
    ssize_t read

    while ((read = getline(&ligne, &lgr, fh) != -1){
      /* ce que vous voulez */
    }
    free(ligne);
  \end{Verbatim}  

  \centerline{\bf\large Pour savoir programmer en C,}
  \centerline{\bf\large il faut programmer en C.}
  \begin{itemize}[noitemsep]
  \item Entraînez-vous, faites tous les TP et (mini--)projets.
  \item Maîtrisez vos outils: [c]make, valgrind, asan, gdb.
  \item Lisez les messages d'erreur pour les comprendre.
  \item Les logiciels libres comme terrain d'entraînement.
  \item Le C sous Windows, c'est encore plus dur.
  \end{itemize}

  \Subsection{Quelques bons livres sur le C:}
  \begin{itemize}[noitemsep,nolistsep]
  \item \href{https://fr.wikibooks.org/wiki/Programmation_C}{\tt fr.wikibooks.org/wiki/Programmation\_C}
  \item \href{https://en.wikibooks.org/wiki/C_Programming}{\tt
      en.wikibooks.org/wiki/C\_Programming}
     
  \item
    \href{http://icube-icps.unistra.fr/index.php/File:ModernC.pdf}
      {{\it Modern C}, livre libre par Jens Gustedt.}
  \item
    \href{https://openclassrooms.com/courses/apprenez-a-programmer-en-c}
    {{\it Apprenez à programmer en C} sur OpenClassRoom.}
  \end{itemize}

  \bigskip %
  Merci d'aider à améliorer les ressources libres, et de m'indiquer
  comment améliorer cette feuille.  %
  \vfill

  ~
  \columnbreak %%%%%%%%%%%%%%%%%%%%%%%%%%%%%%%%%%%%%%%%%%%%%%%%%%%%%%%%
  \VerbatimInput[label=main.c,frame=single,numbers=left,numbersep=2pt]
    {main.c}

  \vfill
  \VerbatimInput[label=point.h,frame=single,numbers=left,numbersep=2pt]
    {point.h}

  \vfill
  \VerbatimInput[label=Makefile,frame=single,numbers=left,numbersep=2pt]
    {Makefile.point}
  
  \vfill
  \VerbatimInput[label=CMakeLists.txt (pour cmake),frame=single,numbers=left,numbersep=2pt]
    {CMakeLists.txt}

  \columnbreak %%%%%%%%%%%%%%%%%%%%%%%%%%%%%%%%%%%%%%%%%%%%%%%%%%%%%%%%
  \VerbatimInput[label=point.c,frame=single,numbers=left,numbersep=2pt]
    {point.c}

  \VerbatimInput[label=Exemple de session avec make,frame=single, numbers=left,
    numbersep=2pt]{shell.make} 
  \VerbatimInput[label=Exemple de session avec cmake,frame=single, numbers=left,
    numbersep=2pt]{shell.cmake} 
\end{multicols} 
\end{document}

%%% Local Variables:
%%% coding: utf-8
