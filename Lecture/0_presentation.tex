\setcounter{part}{-1}
\part{Introduction}
% Corrige l'absence de la page de garde pour cette partie
\makeatletter\beamer@partstartpage=1\makeatother
\section{Introduction}
\begin{frame}{Introduction}
  \begin{block}{Course Goals}
    \begin{itemize}
    \item Help you mastering your thid programming language
      \begin{itemize}
      \item Basics about the syntax
      \item Caveats (of memory management, amongst other)
      \item Get some good style
      \end{itemize}
    \item Help you mastering your Linux box (or any other UNIX-based one)
      \begin{itemize}
      \item Fluent use of the terminal
      \item Non-trivial command lines
      \item Simple scripts
      \end{itemize}

    \end{itemize}
  \end{block}

  \begin{block}{Prerequisite}
    \begin{itemize}
    \item Algorithmic Background: you cannot program without that
    \item Scala/Java Programming: we won't learn to program, but how to write it in C
    \end{itemize}
  \end{block}

  \begin{block}{Course Context at Telecom Nancy}
    \begin{itemize}
    \item Part of Programing Track (courses PPP, TOP, POO, SD, CSH)
    \item Starts a new track on Operating System (courses CSH, RS, RSA)
    \end{itemize}
  \end{block}
\end{frame}
%%%%%%%%%%%%%%%%%%%%%%%%%%%%%%%%%%%%%%%%%%%%%%%%%%%%%%%%%%%%%%%%%%%%%%%%%%%%%%
\begin{frame}{Administrativae}
  \begin{block}{Module Time Table}
    \begin{itemize}
    \item Three lectures
    \item 7 practical labs + 3 repetition sessions (+ exam): The C language 
    \item 6 practical labs + 2 small group lectures (+ exam): Shell Scripting
    \end{itemize}
  \end{block}
  \begin{block}{Evaluation}
    \begin{itemize}
    \item \structure{Test on table \textit{(partiel)} on C language}
      \begin{itemize}
      \item \structure{What:} Content of lectures and labs (of course)
      \item \structure{When:} someday in march (check ADE agenda)
      \item \structure{Allowed material during test:} one A4 sheet of paper only
        \begin{itemize}
        \item Hand-written (not typed)
        \item From you (no photocopy)
        \end{itemize}
      \end{itemize}
    \item \structure{Homework:} Do whatever you want (in C)
    \item \structure{Test on table on Shell Scripting}\\
      \begin{itemize}
      \item \structure{When:} someday in may (check ADE agenda)
      \item (Ask Suzanne Collin for details)
      \end{itemize}
    \end{itemize}
  \end{block}
\end{frame}
%%%%%%%%%%%%%%%%%%%%%%%%%%%%%%%%%%%%%%%%%%%%%%%%%%%%%%%%%%%%%%%%%%%%%%%%%%%%%%
\begin{frame}{About me}
  \begin{block}{Martin Quinson}
    \begin{itemize}
    \item \structure{Study:} Universit\'e de Saint \'Etienne, France
    \item \structure{PhD:} Grids and HPC in 2003 (team Graal of INRIA /
      ENS-Lyon, France)
    \item \structure{Since 2005:}
      \begin{itemize}
      \item Assistant professor at Telecom Nancy (Université de Lorraine)
        % ESIAL (Univ. Henri Poincar\'e--Nancy I, France) 
      \item Researcher of Algorille team (soon of Veridis) of LORIA/Inria
      \end{itemize}
    \item \structure{Research interests:}
      \begin{itemize}
      \item \structure{Context:} Distributed Systems (Grids, HPC, Clusters)
      \item \structure{Main:} Simulation of Distributed Applications (SimGrid
        project) 
      \item \structure{Others:} Experimental Methodology, Model-Checking, ...
      \end{itemize}
    \item \structure{Teaching duties:}
      \begin{itemize}
%      \item Responsible of first year cursus at ESIAL
      \item[1A:] \structure{PPP:} introduction to Java;
        \structure{TOP:} Technics and tOols for Programming;\\
        \structure{CSH:} C as Second Language (and Shell)
      \item[2A:] \structure{RS:} System Programming (and Networking)
      \item[3A:] Peer-to-Peer Systems and Advanced Distributed Algorithms
        (master)
      \end{itemize}

    \item \structure{More infos:}
      \begin{itemize}
      \item \url{http://www.loria.fr/~quinson} ~~
      (\url{Martin.Quinson@loria.fr})
      \end{itemize}
    \end{itemize}
  \end{block}
\end{frame}
%%%%%%%%%%%%%%%%%%%%%%%%%%%%%%%%%%%%%%%%%%%%%%%%%%%%%%%%%%%%%%%%%%%%%%%%%%%%%%
\section{References}
\begin{frame}{References: Courses on Internet}
 % \begin{block}{Courses on Internet}
    \begin{itemize}
    \item \structure{Introduction to Systems Programming} (C. Grothoff)\\
         {\small
      C covered, but not only.\\
      \url{http://grothoff.org/christian/teaching/2009/2355/}} 
      
    \item \structure{C / Shell} (A. Crouzil, J.D. Durou et Ph. Joly;
      U. Paul Sabatier, Toulouse)\\
      {\small 
        Good coverage of the whole module (in French).\\
        \url{http://www.irit.fr/ACTIVITES/EQ_TCI/ENSEIGNEMENT/CetSHELL/}}

    \item \structure{Support de Cours de Langage C} (Christian Bac; Telecom SudParis)\\
      {\small
        The C Language (in French).\\
        \url{http://picolibre.int-evry.fr/projects/svn/coursc/}}
    \end{itemize}
    % \end{block}\vspace{-\baselineskip}
\end{frame}
%%%%%%%%%%%%%%%%%%%%%%%%%%%%%%%%%%%%%%%%%%%%%%%%%%%%%%%%%%%%%%%%%%%
% \begin{frame}{References: Books}
%   \begin{itemize}
%   \item Coulouris, Jean et Kindberg. \structure{Distributed Systems:
%       Concepts and Design}. %\\
% %    Addison-Wesley, ISBN 0201-619-180 (ed. 3) et ISBN 0321263545 (ed. 4).
%   \item Tannenbaum, Steen. \structure{Distributed Systems: Principles and
%       Paradigms}.
%   \item V. K. Garg. \structure{Elements of Distributed Computing.}
%   \item Ralf Steinmetz, Klaus Wehrle (Eds): \structure{Peer-to-Peer
%       Systems and Applications.}
%     \url{http://www.peer-to-peer.info/}
%   \end{itemize}
%   \bigskip

%   \centerline{%
% %    \includegraphics[height=7\baselineskip]{img/cover-coulouris3.jpg}%
% %    ~
% %    \includegraphics[height=7\baselineskip]{img/cover-coulouris4.jpg}%
%     ~~~~
% %    \includegraphics[height=7\baselineskip]{img/cover-tannenbaum.jpg}%
%     ~~~~
% %    \includegraphics[height=7\baselineskip]{img/cover-garg.jpg}%
%     ~~~~
% %    \includegraphics[height=7\baselineskip]{img/cover-p2p.jpg}%
%   }
% \end{frame}
%%%%%%%%%%%%%%%%%%%%%%%%%%%%%%%%%%%%%%%%%%%%%%%%%%%%%%%%%%%%%%%%%%%%%%%%%%%%%
\section{Table of contents}
\begin{frame}[squeeze]{Table of Contents}
  \begin{itemize}
  \item \structure{\large Introduction and Generalities}
    \begin{itemize}
    \item Introduction; Motivation; History.
    \end{itemize}
    \medskip
  \item[\numberedball{1}] \structure{\large Part I: C as Second Language}
    \begin{itemize}
    \item \structure{Syntax and Basic usage}
      \begin{itemize}
      \item Introduction; First C program and compilation; Syntax, printf; C
        \textit{vs.} Java.
      \end{itemize}
      
    \item \structure{Memory Management in C}
      \begin{itemize}
      \item Variable visibility, storage class; Malloc and friends; Debugging
        problems. 
      \end{itemize}
      
    \item \structure{Advanced C Topics}
      \begin{itemize}
      \item Modularity in C; Makefile; Performance tuning; Game programming.
      \end{itemize}
    \end{itemize}
    \medskip
  \item[\numberedball{2}] \structure{\large Part II: Shell Scripting}
    \begin{itemize}
    \item \structure{Low Script-fu knowledge}
      \begin{itemize}
      \item Introduction; First shell ``scripts''; Redirecting I/O \& Pipes;
        basic commands.
      \end{itemize}
    \item \structure{Medium Script-fu knowledge}
      \begin{itemize}
      \item More Syntax for Advanced Scripts; Not so basic commands.
      \end{itemize}
    \item \structure{Advanced Script-fu knowledge}
      \begin{itemize}
      \item Shell functions; Variable Substitutions; Sub-shells; Arrays.  
      \end{itemize}

    \end{itemize}
  \end{itemize}    
\end{frame}