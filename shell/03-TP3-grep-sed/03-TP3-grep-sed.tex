%----------------------------------------------------------------------------
%                 
%                        
%
%                                    TP3 sh 
%                         �criture de script -  grep et sed
%                           
%------------------------------------------------------------------------------

\documentclass{article}
\usepackage[french,english]{babel}
\usepackage[latin1]{inputenc}
\usepackage{alltt}
\usepackage{t1enc}
\usepackage{graphicx}
\usepackage{verbatim}

%------------

 
\setlength{\oddsidemargin}{2mm}
\setlength{\oddsidemargin}{2mm}
\setlength{\evensidemargin}{2mm}
\setlength{\textwidth}{154mm}
\setlength{\textheight}{230mm}
\setlength{\topmargin}{-3mm}
\setlength{\parskip}{1mm}
\setlength{\unitlength}{1mm}
\setlength{\parindent}{0mm}


\begin{document}


\parbox{3cm}
{\includegraphics[width=3cm]{/home/collin/svn-collin/ENSEIGNEMENT/Telecom_Nancy_logo_Q_CS4.eps}}
% \ \hfill \


\begin{center}
\fbox{
{\Large\sffamily\bfseries  TP shell (3)\,: grep, sed\ldots(suite)}
}
\end{center}

% une macro de double trait 

\long\def\dtrait{\vspace{-1.6 cm}
\rule[-15 mm]{\textwidth}{0.5 mm}\\
\rule[15 mm]{\textwidth}{0.2mm}
\vspace{-2 cm}}

% une macro de r�ponse

\long\def\reponse{\hspace{5cm} \hrulefill}

\dtrait       

\section*{\sffamily\bfseries EXERCICE 1\,: commandes diverses. } 
%--------------------------------------------------------------

Ecrire une commande {\tt audit}  qui regarde dans votre {\it home} les
fichiers ex�cutables par tous les utilisateurs et les affiche
(utilisez {\tt find} et {\tt grep}).

~\\
Ecrire une commande  {\tt vire-ligne-vide} permettant de lister le
contenu  d'un 
fichier pass� en argument de la commande, sans les lignes vides et en
les num�rotant. 


~\\
Ecrire une commande  {\tt grep-rec} qui recherche dans l'arborescence
situ�e sous le premier argument, toutes les lignes qui v�rifient
l'expression situ�e en second argument, dans le fichier pass� en
troisi�me argument.\\
Par exemple\,: 
\begin{verbatim} 
  grep-rec . "^#"  "*.c"
\end{verbatim}


Indication\,: utiliser {\tt xargs}.


\section*{\sffamily\bfseries EXERCICE 2\,: affiche-session.}
%---------------------------------------------------------

Ecrire un script {\tt affiche-session} qui affiche la liste de tous
les fichiers modifi�s dans la journ�e.
~\\
On peut �crire cette commande � l'aide de {\em find}, {\em cut} et
aussi {\em sed}.
Ecrivez dans un premier temps une version avec {\em cut}, puis une
version avec {\sffamily sed}. 


\section*{\sffamily\bfseries EXERCICE 3\,: affiche} 
%---------------------------------------------------------
\begin{enumerate}
\item �crire une commande {\tt affiche} qui liste les noms des fichiers
  dans le r�pertoire courant qui sont tels que :
\begin{itemize}
\item ils poss�dent le suffixe {\tt .c}
\item ils contiennent la ligne :

\begin{center}
{\tt \#include $<$stdio.h$>$}
\end{center}
\end{itemize}

% \item Am�liorer la commande pour autoriser les espaces et les tabulations
%  entre toutes les entit�s.

% \begin{center} \begin{tabular}{cccccccccc}
% & {\tt \#} && {\tt include} && {\tt $<$} && {\tt stdio.h} && {\tt $>$} \\
% $\uparrow$ && $\uparrow$  && $\uparrow$  && $\uparrow$ && $\uparrow$ \\
% espaces && ici && ici && ici && ici \\
% tabulations
% \end{tabular} \end{center}

\item Am�liorer la commande afin de v�rifier que la ligne ~~{\tt \#include
    $<$stdio.h$>$} n'est pr�sente qu'une seule fois et exclusivement sur la
  premi�re ligne du fichier.
\end{enumerate}


\section*{\sffamily\bfseries EXERCICE 4\,: replace} 
%---------------------------------------------------------

Ecrire une commande {\tt replace} qui est  un {\em sed}
simplifi�, � utiliser dans les cas simples.
La syntaxe de cette commande est la suivante\,:  {\tt replace chaine1
  chaine2 fich}.
Elle a pour effet de  remplacer toute occurence de  {\tt chaine1} par
{\tt chaine2} dans le fichier  {\tt fich}.

Faites en sorte que cette commande fonctionne aussi dans le cas d'une
liste de fichiers  (par exemple\,: {\tt replace chaine1
  chaine2 '*.txt'}).


\section*{\sffamily\bfseries EXERCICE 5\,: Commande chsuff.} 
%---------------------------------------------------------

Dans la page {\em man bash} on trouve l'extrait suivant qui vous
sera utile d'appliquer dans l'exercice qui suit (plut�t que d'utiliser
{\em sed})\,:

\dtrait
{\small
\begin{verbatim}
Given
    a=/a/b/c/d
    b=b.xxx

    csh            bash        result
    ---            ----        ------
    $a:h            ${a%/*}        /a/b/c
    $a:t            ${a##*/}       d
    $b:r            ${b%.*}        b
    $b:e            ${b##*.}       xxx 
\end{verbatim}
}
\dtrait

\begin{enumerate}

\item �crire une commande {\tt chsuff} � deux arguments, le premier
est une cha�ne de caract�res repr�sentant un suffixe ou le
symbole\,{\tt -}, le deuxi�me est un nom de fichier ou de r�pertoire.

L'ex�cution de la commande :
\begin{center}
{\tt chsuff cha�ne nom}
\end{center}
\begin{tabbing}
a pour \= effet : ~~~~~ \= $\bullet$ de remplacer le suffixe de {\tt
  $<$nom$>$} par {\tt $<$cha�ne$>$}, si {\tt $<$nom$>$} comporte un suffixe,\\
\> ou \> $\bullet$ de renommer l'objet par {\tt $<$nom$>$.$<$cha�ne$>$} sinon.
\end{tabbing}

L'ex�cution de la commande :
\begin{center}
{\tt chsuff - nom}
\end{center}
a pour effet de supprimer le suffixe de {\tt $<$nom$>$}, s'il existe.

Exemples d'utilisation :
\begin{tabbing}
coucou \= \kill
\> {\tt chsuff old prog.c} ~~~~~ \= renomme {\tt prog.c} en {\tt prog.old}
~~~~ \= {\it  (changement de suffixe)} \\
\> {\tt chsuff xx prog} \> renomme {\tt  prog} en {\tt prog.xx} \> {\it
  (ajout d'un suffixe)} \\
\> {\tt chsuff - prog.old} \> renomme {\tt prog.old} en {\tt prog} \> {\it
  (suppression du suffixe)} \\
\> {\tt chsuff - fichier} \> ne doit rien faire...
\end{tabbing}

\item Reprendre la commande de fa�on � ce qu'elle s'applique � une
liste de noms de fichiers ou de r�pertoires.

\item {\em Questions subsidiaires.}\\
Quand on demande l'ex�cution de la commande :
\begin{center}
{\tt chsuff - fich}
\end{center}
il est possible qu'un message d'erreur de la commande {\tt mv} s'affiche :
\begin{center}
{\tt mv: fich and fich are identical}
\end{center}

Modifier votre {\it script} pour faire �crire le message par
votre {\it script} et non pas par la commande {\tt mv}.

\item Que se passe-t'il quand vous demandez la suppression du suffixe d'un
  fichier commen�ant par le caract�re point ?  Adaptez votre {\it script}
  en cons�quence.

\end{enumerate}

%-------------------------------------------------------------------------
\end{document}

