\documentclass[10pt, sansserif,
               firstcolor=color1,
               secondcolor=color2,
               logo=logo-TN, 
               footband=bandeau-TN]{TelecomNancy}
\usepackage[utf8]{inputenc}
\author{AZE}
%\usepackage{esial}\CSH\1A
%\usepackage[correction]{esial}\CSH\1A
\newcommand{\cd}[1]{\medskip\noindent\file{\null\hspace{-1em}[#1] }}
\newcommand{\touche}[1]{\hbox{$<$#1$>$}}
\newcommand{\ctrl}[1]{\touche{ctrl-#1}}
\newcommand{\tab}{\touche{TAB}}
\sloppy
\usepackage{textcomp,amstext}

\newcommand{\unix}[1]{\hspace*{2cm}{\tt #1}}
\newcommand{\fich}[1]{{\bf \em #1}}

\newcommand{\BoxRep}{\ifcorrection{\boxtimes}{\Box}}
\newenvironment{Reponse}{}{}
\let\Reponse=\comment  \let\endReponse=\endcomment
\newcommand{\run}[1]{\fbox{\texttt{#1}}}
\newcommand{\cmd}[1]{\texttt{#1}}
\newcommand{\file}[1]{{\bf \em #1}}
\newcommand{\result}[1]{\texttt{#1}}
\usepackage{fancyvrb}
\fvset{frame=single, numbersep=5pt}

\coursename{Première année}
\studentlevel{Langage C et Shell}
\doctitle{Devoir surveillé de C, 13 mai 2014~~~}
\definecolor{color1}{HTML}{6C2466}
\definecolor{color2}{HTML}{EF8A26}

\begin{document}
\globalinstructions[\vspace{-\baselineskip}~]{ %
  \hspace{-2em} Tous documents interdits à l'exception d'une page A4 recto,
  manuscrite de votre main. La correction tiendra compte de la qualité de
  l'argumentaire et de la présentation. Le barème indiqué est approximatif.}


\nextExercise[Compter les voyelles (chaînes de caractères et fichiers -- 4pts)]{
  On souhaite écrire un programme affichant le décompte des voyelles présentes
  dans une chaîne de caractères donnée.}

\nextQuestion{Écrivez un programme complet prenant une chaîne de caractères à
  analyser en argument de la ligne de commande, de la manière suivante (2pts).}

\begin{Verbatim}
  $ ./compte-voyelles "Bonjour. Bon courage pour ce partiel."
  Nombre de a: 2
  Nombre de e: 3
  Nombre de i: 1
  Nombre de o: 5
  Nombre de u: 3
  Nombre de y: 0
\end{Verbatim}
%$

\nextQuestion{Modifiez votre programme afin qu'il lise le texte à analyser
  depuis un fichier dont le nom est passé en argument de la ligne de commandes
  (2pt).}

%%%%%%%%%%%%%%%%%%%%%%%%%%%

\nextExercise[Gestion de parc locatif (programmation structurée, POO -- 6pts)]{
  Pour gagner en efficience, un loueur de voitures décide de concevoir son
  propre progiciel de gestion intégré (ERP) pour gérer son parc. Chaque voiture
  est définie par sa catégorie (Citadine, Berline, Familiale, Utilitaire), sa
  plaque d'immatriculation et son prix par jour. Chaque voiture peut également
  être actuellement disponible, ou bien déjà louée.}

\nextQuestion{Écrivez une structure de données \texttt{voiture\_t} contenant
  toutes les informations relatives à une voiture donnée. Vous suivrez
  l'approche plutôt orientée objet vue en cours. (\textonehalf pt)}

\nextQuestion{Écrivez une fonction \texttt{voiture\_new()}, prenant en arguments
  les informations initiales, et retournant une référence à un «objet»,
  c'est-à-dire un pointeur sur une nouvelle structure
  \texttt{voiture\_t} (1pt).}
\nextQuestion{Écrivez une fonction \texttt{voiture\_free()}, prenant une référence
  à une voiture pour la détruire en libérant la mémoire qu'elle occupait
  (\textonehalf pt).}

\nextQuestion{Écrivez une fonction \texttt{voiture\_est\_libre()} retournant si
  oui ou non la voiture en paramètre est libre. Écrivez une fonction
  \texttt{voiture\_loue()} qui s'assure que la voiture en paramètre n'est plus
  libre (\textonehalf pt).}

\nextQuestion{Écrivez la déclaration d'un tableau nommé \texttt{voitures},
  permettant de stocker les informations relatives à toutes les voitures du
  loueur. Une variable \texttt{nb\_voitures} stockera la taille de ce tableau
  (\textonehalf pt).}

\nextQuestion{Écrivez une fonction \texttt{voiture\_achete()} créant une
  nouvelle voiture grâce à \texttt{voiture\_new()} ci-dessus et qui l'ajoute au
  tableau. Comment être sûr que le tableau est assez grand ? (2pt)}

\nextQuestion{Écrivez une fonction qui étant donné une catégorie de voitures,
  retourne un pointeur vers la première voiture de cette catégorie qui ne soit
  pas encore louée, ou NULL si aucune n'est disponible (1pt).}

%%%%%%%%%%%%%%%%%%%%%%%%%%%%
\newpage\fancyhead{}
\null\vspace{-7\baselineskip}
\nextExercise[Questions de cours (4pts)]{}

\nextQuestion{Que signifient les messages d'erreur suivants? Quand arrivent-ils,
  et comment les corriger? (2pts).}
\begin{itemize}\setlength{\itemsep}{0pt}\setlength{\parsep}{0pt}
\item[(a)] \textit{Control reaches end of non void function}
\item[(b)] \textit{Implicit declaration of function toto}
\end{itemize}\vspace{-\baselineskip}

\nextQuestion{Expliquez les problèmes posés par ces morceaux de programmes, puis
  proposez des solutions (2pts).}

\noindent 
\begin{minipage}[t]{.23\linewidth}
  \begin{Verbatim}[numbers=left,label=Programme 3.2a]
char *truc;
*truc = 'x'; 
  \end{Verbatim}
\end{minipage}\hfill
\begin{minipage}[t]{.3\linewidth}
  \begin{Verbatim}[numbers=left,label=Programme 3.2b]
char *truc = "constant";
truc[0] = 'x'; 
free(truc);          
  \end{Verbatim}
\end{minipage}\hfill
\begin{minipage}[t]{.37\linewidth}
  \begin{Verbatim}[numbers=right,label=Programme 3.2c]
int *make_buff(int a) {
  int buff[SIZE], cpt;
  for (cpt=0; cpt<SIZE; cpt++) 
    buff[cpt] = a;
  return buff;
}    
  \end{Verbatim}
\end{minipage}

%%%%%%%%%%%%%%%%%%%%%%%%

\vspace{-\baselineskip}
\nextExercise[Lecture de code (6pts)]{Indiquez ce qu'affiche l'exécution des programmes
  C suivants, qui compilent et s'exécutent sans erreur.}

\bigskip\noindent
  \begin{minipage}[t]{.35\linewidth}
  \begin{Verbatim}[label=Programme 4.1 (début),numbers=left]
#include <stdlib.h>
#include <stdio.h>
void f1 ( int a, int *b ) {
  a = *b;
}
void f2 ( int *b, int c ) {
  *b = c;
}
void f3 ( int *a, int c ) {
  f4 ( &a, c );
}
void f4 ( int **b, int a ) {
  **b = a;
}
  \end{Verbatim}
  \end{minipage}~~~~
  \begin{minipage}[t]{.6\linewidth}
  \begin{Verbatim}[label=Programme 4.1 (fin),numbers=right,firstnumber=last]
int main () {
  int x = 5, y = 7, z = 9;
  f1 ( x, &y );
  printf ( "x = %d, y = %d, z = %d\n", x, y, z );
  f2 ( &x, y );
  printf ( "x = %d, y = %d, z = %d\n", x, y, z );
  f3 ( &y, z );
  printf ( "x = %d, y = %d, z = %d\n", x, y, z );
  return 0;
}    
  \end{Verbatim}    
  \end{minipage}

%%%%%%%%%%%%%%%%%%%%%%%%%%

  
\bigskip
\noindent
\VerbatimInput[numbers=left, label=Programme 4.2]{2014/lecture2.c}
\end{document}

%\chead{}
\lhead{\LARGE NOM:}
\chead{\LARGE PRÉNOM:}
\newpage
\begin{Exercice}


\Question \textbf{(3 pts)}  Qu'affiche ce programme lors de son exécution?
\begin{Reponse}
\noindent x = 5, y = 7, z = 9\\
x = 7, y = 7, z = 9\\
x = 7, y = 9, z = 9

(-1/2 par erreur, mais l'exo est largement sur-noté. Il vaut pas 3 pts)
\end{Reponse}
\end{Exercice}

\begin{Exercice} \textbf{QCM}
Répondez sur la feuille fournie. Il peut y avoir plusieurs cases valides par
ligne; les réponses fausses seront pénalisées.

\Question \textbf{(2 pts)} Quel est le type de chacune des variables dans cet
extrait de programme?

\noindent\hspace{-.8em}\begin{minipage}{.28\linewidth}
\begin{Verbatim}
int *a,b;
char **c, *d[12];

typedef struct {
  char *marque;
  int nb_places;
  float consomation;
} voiture_t, *voiture; 

voiture_t e,f[12];
voiture   g,h[12];
\end{Verbatim}  
\end{minipage}~\begin{minipage}{.7\linewidth}
\begin{tabular}{|l|c|c|c|c|c|c|}\hline
  variable & entier & flottant & structure & pointeur & tableau & 
  $\left(\!\!\begin{array}{c}
    \text{écriture}\\
    \text{invalide}
  \end{array}\!\!\right)$ \\\hline
  a & $\Box$ & $\Box$ & $\Box$ & $\BoxRep$ & $\Box$ & $\Box$\\\hline
  b & $\BoxRep$ & $\Box$ & $\Box$ & $\Box$ & $\Box$ & $\Box$\\\hline
  c & $\Box$ & $\Box$ & $\Box$ & $\BoxRep$ & $\Box$ & $\Box$\\\hline
  d & $\Box$ & $\Box$ & $\Box$ & $\Box$ & $\BoxRep$ & $\Box$\\\hline
  e & $\Box$ & $\Box$ & $\BoxRep$ & $\Box$ & $\Box$ & $\Box$\\\hline
  e.nb\_places    & $\BoxRep$ & $\Box$ & $\Box$ & $\Box$ & $\Box$ & $\Box$\\\hline
  e-$>$nb\_places & $\Box$ & $\Box$ & $\Box$ & $\Box$ & $\Box$ & $\BoxRep$\\\hline
  f & $\Box$ & $\Box$ & $\Box$ & $\Box$ & $\BoxRep$ & $\Box$\\\hline
  f.nb\_places    & $\Box$ & $\Box$ & $\Box$ & $\Box$ & $\Box$ & $\BoxRep$\\\hline
  f-$>$nb\_places & $\Box$ & $\Box$ & $\Box$ & $\Box$ & $\Box$ & $\BoxRep$\\\hline
  f[0].nb\_places    & $\BoxRep$ & $\Box$ & $\Box$ & $\Box$ & $\Box$ & $\Box$\\\hline
  f[0]-$>$nb\_places & $\Box$ & $\Box$ & $\Box$ & $\Box$ & $\Box$ & $\BoxRep$\\\hline
  g & $\Box$ & $\Box$ & $\Box$ & $\BoxRep$ & $\Box$ & $\Box$\\\hline
  g.nb\_places    & $\Box$ & $\Box$ & $\Box$ & $\Box$ & $\Box$ & $\BoxRep$\\\hline
  g-$>$nb\_places & $\BoxRep$ & $\Box$ & $\Box$ & $\Box$ & $\Box$ & $\Box$\\\hline
  h & $\Box$ & $\Box$ & $\Box$ & $\Box$ & $\BoxRep$ & $\Box$\\\hline
  h.nb\_places    & $\Box$ & $\Box$ & $\Box$ & $\Box$ & $\Box$ & $\BoxRep$\\\hline
  h-$>$nb\_places & $\Box$ & $\Box$ & $\Box$ & $\Box$ & $\Box$ & $\BoxRep$\\\hline
  h[0].nb\_places    & $\Box$ & $\Box$ & $\Box$ & $\Box$ & $\Box$ & $\BoxRep$\\\hline
  h[0]-$>$nb\_places & $\BoxRep$ & $\Box$ & $\Box$ & $\Box$ & $\Box$ & $\Box$\\\hline
\end{tabular}
\end{minipage}
\begin{Reponse}
  \begin{itemize}
  \item +0.1 par réponse juste
  \item -0.1 par réponse fausse
  \item 0 si blanc ou une réponse juste et une réponse fausse sur la ligne
  \item arrondi au quart supérieur
  \end{itemize}
\end{Reponse}

\Question \textbf{(1 pt)}
\begin{itemize}
\item[$\bullet$] Si dans un fichier \textit{makefile} on trouve la ligne suivante :
  \run{toto: tutu}
  \begin{itemize}
  \item[$\BoxRep$] cela veut dire qu'il faut reconstruire \texttt{toto} chaque
    fois que \texttt{tutu} change
  \item[$\Box$] cela veut dire qu'il faut reconstruire \texttt{tutu} chaque
    fois que \texttt{toto} change
  \end{itemize}
\item[$\bullet$] Dans un fichier d'entêtes (.h), on peut trouver:
  \begin{itemize}
  \item[$\Box$] Des définitions de fonctions
  \item[$\BoxRep$] Des prototypes de fonctions
  \item[$\BoxRep$] Des définitions de types
  \end{itemize}
\item[$\bullet$] L'expression \run{if [ -x toto -a -d toto ]} teste si:
  \begin{itemize}
  \item[$\Box$] on a le droit d'exécuter et d'effacer \texttt{toto}
  \item[$\Box$] on a le droit d'exécuter \texttt{toto} et si sa taille est non nulle
  \item[$\Box$] \texttt{toto} est un répertoire dont on peut lire le contenu
  \item[$\BoxRep$] \texttt{toto} est un répertoire dans lequel on peut entrer
  \end{itemize}
\end{itemize}
\begin{Reponse}
  \begin{itemize}
  \item +0.25 par case cochée à juste titre
  \item -0.25 par case cochée à tord
  \item 0 si blanc
  \end{itemize}
\end{Reponse}

\end{Exercice}

      
\end{document}



% LocalWords:  Makefile Laurel strings
%%% Local Variables:
%%% coding: utf-8
