\documentclass[10pt]{article}
\usepackage[utf8]{inputenc}
\usepackage{esial}\CSH\1A
%\usepackage[correction]{esial}\CSH\1A
\newcommand{\cd}[1]{\medskip\noindent\file{\null\hspace{-1em}[#1] }}
\newcommand{\touche}[1]{\hbox{$<$#1$>$}}
\newcommand{\ctrl}[1]{\touche{ctrl-#1}}
\newcommand{\tab}{\touche{TAB}}
\sloppy
\usepackage{textcomp,amstext}

\newcommand{\unix}[1]{\hspace*{2cm}{\tt #1}}
\newcommand{\fich}[1]{{\bf \em #1}}

\newcommand{\BoxRep}{\ifcorrection{\boxtimes}{\Box}}

\title{Devoir surveillé du 11 juin 2008 (consolidation)}
\begin{document}
\maketitle
\thispagestyle{empty}

\begin{quote}
  Tous documents interdits. Les exercices sont indépendants. La correction
  tiendra compte de la qualité de la rédaction et de la présentation. Barème
  approximatif.
\end{quote}


\begin{Exercice} \textbf{(6pts)}
  On souhaite écrire une fonction qui compte le nombre d'occurrence de
  caractères dans une chaîne donnée. 
  
  \Question (2pts) Écrivez la fonction \run{int compte\_un(char
    *chaine, char car)} comptant le nombre d'occurrences du caractère passé en
  deuxième argument dans la chaîne passée en premier argument. Ainsi,
  \run{compte\_un("bonjour",b)} doit retourner 1 tandis que 
  \run{compte\_un("bonjour",o)} doit retourner 2.

\begin{Reponse}
  \begin{Verbatim}
int compte_un(char *chaine, char car) {
  int res = 0;
  char *p = chaine;
  while (*p != '\0') {
    if (*p == car)
      res++;
    p++;
  }
  return res;
}
\end{Verbatim}
\end{Reponse}

\Question (4pts) Écrivez la fonction \run{int *compte\_tous(char *chaine)} qui
retourne un tableau de 26 entiers (que vous devrez allouer), chacun indiquant
le nombre d'occurrences de la lettre correspondante.

\begin{Reponse}
\begin{Verbatim}
int *compte_tous(char *chaine) {
  int* res = malloc(sizeof(int)*26);
  int i;
  char *p = chaine;
  for (i=0;i<26;i++) {
    res[i] = 0;
  }
  while (*p != '\0') {
    if (*p<='z' && *p >= 'a') {
      res[*p-'a']++;
    }
  }
  return res;
}
  \end{Verbatim}
\end{Reponse}
\end{Exercice}

\medskip
\begin{Exercice} \textbf{(7pts)} Soit un fichier regroupant les résultats de
  l'euro 2008 sous la forme suivante :

  \noindent
  \begin{minipage}{.3\linewidth}
  \begin{Verbatim}
Suisse Rep.Tcheque 0 1
Portugal Turquie 2 0
Rep.Tcheque Portugal 1 3  
Autriche Croatie 0 1
Allemagne Pologne 2 0
Roumanie France 0 0
Pays-Bas Italie 3 0
Espagne Russie 4 1
Grèce Suède 0 2
  \end{Verbatim}    
  \end{minipage}\hfill%
  \begin{minipage}{.68\linewidth}
  Les deux premières colonnes sont les équipes ayant joué le match (dans un
  ordre non spécifié) et les deux suivantes sont les scores respectifs de ces
  équipes. Les colonnes sont séparées par une seule espace.
  \end{minipage}

  \Question (\textonehalf~pt) Écrire une commande shell donnant tous les
  résultats concernant la France.

\begin{Reponse}
  \begin{Verbatim}
grep France fichier    
  \end{Verbatim}
\end{Reponse}

\Question (\textonehalf~pt) Écrire une commande donnant le score du match
entre la France et la Roumanie.

\begin{Reponse}
  \begin{Verbatim}
grep France fichier | grep Roumanie 
  \end{Verbatim}
 On peut pas faire un seul grep (facilement) car on ne sait pas dans quel ordre
 sont listées les équipes.
\end{Reponse}

\Question (2pts) Écrire un script shell listant tous les matches s'étant soldés
par un match nul (même nombre de buts inscrits par chaque équipe).

\begin{Reponse}
  Version inélégante: 
  \begin{Verbatim}
grep "0 0'' fichier; grep "1 1" fichier; grep "2 2" fichier (etc)
  \end{Verbatim}
  Version plus réfléchie:
  \begin{Verbatim}
for i in 0 1 2 3 4 5 6 7 8 9 10 11 12 ; do
  grep "$i $i" fichier
done    
  \end{Verbatim}
  Version propre:
  \begin{Verbatim}
for i in `seq 0 20`; do
  grep "$i $i" fichier
done    
  \end{Verbatim}
\end{Reponse}

\Question (4 pts) Écrire un script shell prenant deux noms d'équipes en
paramètres et indiquant si la première équipe a gagné, si c'est la seconde ou
si le match s'est soldé par un nul.

\begin{Reponse}
  \begin{Verbatim}
match=`grep $1 fichier | grep $2`  # la ligne qui parle de ce match 
                                   # (attention, le portugal a fait 2 matchs !)

un=`echo $match | grep "^$1" | cut -f3 -d' '`   # le champ 3 si $1 est au début
if [ -z $un ] ; then                            # si vide, $1 était en 2ieme
  un=`echo $match | cut -f4 -d' '`              # alors on cherche le champ 4
fi
deux=`echo $match | grep "^$2" | cut -f3 -d' '`       # La même chose avec $2
if [ -z $un ] ; then 
  deux=`echo $match | cut -f4 -d' '`
fi

# On a les deux scores, reste plus qu'à comparer
if [ $un -lt $deux ] ; then 
  echo "C'est la première équipe ($1) qui a gagné"
fi
if [ $un -gt $deux ] ; then 
  echo "C'est la deuxième équipe ($2) qui a gagné"
fi
if [ $un -eq $deux ] ; then 
  echo "C'est un match nul"
fi
  \end{Verbatim}
\end{Reponse}
\end{Exercice}

\smallskip
\begin{Exercice}\textbf{(6pts)}
  \textit{Le ROT13 ("rotate by 13 places") (une variante du chiffre de César)
    est un algorithme très simple de chiffrement de texte. Comme son nom
    l'indique, il s'agit d'un décalage de 13 caractères de chaque lettre du
    texte à chiffrer.
%
  [...]
%
  L'avantage de ROT13, c'est le fait que le décalage soit de 13. L'alphabet
  comporte 26 lettres, et si on applique deux fois de suite le chiffrement, on
  obtient comme résultat le texte en clair. Pour cela on doit considérer
  l'alphabet comme circulaire, c'est-à-dire qu'après la lettre Z on a la lettre
  A, ce qui permet de grandement simplifier son usage et sa programmation
  puisque c'est la même procédure qui est utilisée pour le chiffrement et le
  déchiffrement.} \hfill(D'après Wikipédia).
\medskip

\hspace{-3.5em}\noindent\begin{tabular}{|c|*{26}{c@{$\,$}|}}\hline
  Caractère non-chiffré&A&B&C&D&E&F&G&H&I&J&K&L&M&N&O&P&Q&R&S&T&U&V&W&X&Y&Z\\
  \hline
  Caractère chiffré&N&O&P&Q&R&S&T&U&V&W&X&Y&Z&A&B&C&D&E&F&G&H&I&J&K&L&M\\
  \hline
\end{tabular}

\Question (3pts) Écrivez une fonction \run{void rot13(char *chaine)} appliquant
l'algorithme ROT13 à la chaîne passée en argument (modifiez la chaîne en place
pour plus de simplicité). 

\begin{Reponse}
  \begin{Verbatim}
void rot13(char *chaine) {
  char *p=chaine;
  while (*p) { // jusqu'au \0 final
    if (*p >= 'a' && *p <= 'z') // on ne change que les lettres
      *p = ((*p - 'a' + 13) % 26) + 'a';
      // on commence par remettre la lettre courante sur [0;25] avec "- 'a'"
      // on applique ROT13 avec "(X +13) % 26"
      // on retourne de l'intervale [0;25] à ['a';'z'] avec "+ 'a'" à la fin
      // (les lettres sont des entiers en C; leur valeur numérique est sans intéret)
    p++; // on regarde la lettre suivante
  }
}    
  \end{Verbatim}
\end{Reponse}

\Question (3pts) Écrivez un programme \texttt{rot13} prenant deux arguments. Le
premier est le nom d'un fichier à crypter par ROT13, et le second est le nom du
fichier dans lequel écrire le résultat.

\begin{Reponse}
  \begin{Verbatim}
int main(int argc, char *argv[]) {
  if (argc != 3) {
    fprintf(stderr, "Usage: rot13 input output\n");
    return 1;
  }
  FILE *in = fopen(argv[1], "r");
  FILE *out= fopen(argv[2], "w");
  if (in == NULL || out == NULL) {
    fprintf(stderr, "Erreur à l'ouverture des fichiers\n");
    return 1;
  }
  char * ligne = NULL;
  size_t longueur = 0;

  while (getline(&ligne, &long, in) != -1) {
    rot13(ligne);              // on utilise la fonction de la question d'avant
    fprintf(out, "%s", ligne); // on écrit le résultat dans le fichier de sortie
    // note: ne pas passer ligne directement en 2ieme argument de fprintf 
    // car si y'a '%s' dedans par hasard, on va avoir un gros probleme
  }

  if (ligne)
    free(ligne);
  fclose(in);
  fclose(out);
}    
  \end{Verbatim}
\end{Reponse}

\medskip\noindent \textit{Indication}: la fonction \run{int getline(char
  **ligne\_ptr,int *lgr\_ptr, FILE *flux)} permet de lire un fichier
ligne à ligne comme le montre l'extrait de code suivant:

\noindent\begin{minipage}{.5\linewidth}
\begin{Verbatim}
char * ligne = NULL;
size_t longueur = 0;

while (getline(&ligne, &long, in) != -1) {
  // utiliser ligne
}

// libérer ligne
\end{Verbatim}
  
\end{minipage}~~\begin{minipage}{.5\linewidth}
Si \texttt{ligne} == NULL, la fonction alloue automatiquement un espace
mémoire avec \texttt{malloc()} et fait pointer \texttt{ligne}
dessus. \texttt{longueur} indique la taille du bloc. Si
\texttt{ligne} != NULL, cela doit pointer sur un bloc de
taille \texttt{longueur}. \texttt{getline} peut alors utiliser cet espace, ou
allouer un bloc plus gros au besoin (auquel cas, \texttt{ligne} et
\texttt{longueur} sont mises à jour).
\end{minipage}

\smallskip Tout cela pour dire qu'il suffit de mettre le code utilisant la
\texttt{ligne} lue à l'intérieur de la boucle \texttt{while}.
\end{Exercice}

\end{document}

% LocalWords:  Makefile strings
%%% Local Variables:
%%% coding: utf-8
