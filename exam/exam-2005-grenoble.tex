\documentclass[10pt]{article}
\usepackage{fancyheadings}
\usepackage[latin1]{inputenc}
\usepackage{verbatim}
\usepackage{graphics}
\usepackage{fullpage}
\setlength{\parindent}{0pt}
\setlength{\parskip}{1em}
\newcommand{\unix}[1]{\hspace*{2cm}{\tt #1}}
\newcommand{\fich}[1]{{\bf \em #1}}

\begin{document}
\thispagestyle{empty}

{\bf Université Joseph Fourier} 
\hfill {\bf UE INF122 } \\
{\bf Licence L1 ``Sciences et Technologies''} \\

\vspace{0.4cm}

\begin{center}
{\Large {\bf Devoir Surveillé du 14 janvier 2005}}\\
\end{center}

\vspace{0.4cm}

Durée 1h30 ; Documents autorisés \\
Les exercices sont indépendants 

\vspace{0.2cm}


\subsection*{Exercice 1}

L'utilisateur {\tt laurel} débute une session de travail en examinant 
le contenu du répertoire courant~: 
\begin{verbatim}
{laurel} 5 >ls -l
total 92
-rw-r--r--   1 laurel dsu-etu     226 Jan  4 10:08 Makefile
drwxr-xr-x   2 laurel dsu-etu     512 Jan  7 07:03 TP1
drwxr-xr-x   2 laurel dsu-etu     512 Jan  7 07:03 TP2
-rw-r--r--   1 laurel dsu-etu     129 Nov 10 17:12 deux.c
-rw-r--r--   1 laurel dsu-etu      12 Nov 11 15:04 deux_entiers
-rw-r--r--   1 laurel dsu-etu     225 Nov 10 17:13 trois.c
-rw-r--r--   1 laurel dsu-etu     173 Jan  7 07:07 installeTP.sh
-rwxr-xr-x   1 laurel dsu-etu    1115 Jan  8 17:04 max2
-rw-r--r--   1 laurel dsu-etu     897 Nov 10 17:14 un.c
-rw-r--r--   1 laurel dsu-etu     154 Dec 27 15:29 y.c
\end{verbatim}

{\bf 1.} Laurel exécute alors différentes commandes qui se terminent
toutes par un message d'erreur.
Expliquez dans chaque cas le sens de ces messages d'erreurs et indiquez
comment remédier au problème.

\begin{enumerate}
\item
\begin{verbatim}
{laurel} 2 >more deux_entiers 
23 12
{laurel} 3 >max2 < resultat > deux_entiers
resultat: No such file or directory.
{laurel} 4 >
\end{verbatim}
\vspace*{1em}
\item
\begin{verbatim}
{laurel} 6 >which installeTP.sh 
installeTP.sh: Command not found.
{laurel} 7 >
\end{verbatim}
\vspace*{1em}
\item
\begin{verbatim}
{laurel} 21 >gcc -o y y.c
Undefined                       first referenced
 symbol                             in file
ploumploum                          /var/tmp//ccMi6qE9.o
ld: fatal: Symbol referencing errors. No output written to y
collect2: ld returned 1 exit status
{laurel} 22 >
\end{verbatim}
\vspace*{1em}
\item
\begin{verbatim}
{laurel} 55 >touch *.c
{laurel} 56 >make
gcc -c un.c
gcc -c deux.c
gcc -c trois.c
gcc -o quatre un.o deux.o trois.o
ld: fatal: symbol `zero' is multiply-defined:
        (file deux.o type=FUNC; file trois.o type=FUNC);
ld: fatal: File processing errors. No output written to quatre
collect2: ld returned 1 exit status
*** Error code 1
make: Fatal error: Command failed for target `quatre'
{laurel} 57 >
\end{verbatim}
\end{enumerate}

{\bf 2.} Ecrivez un contenu possible pour le fichier \fich{Makefile}.

{\bf 3.} L'utilisateur {\tt hardy} a aussi écrit un \fich{Makefile} dans son 
	répertoire personnel. Comment Laurel sait-t-il s'il peut lire le \fich{Makefile}
	de son ami ? Quelle(s) commande(s) pourra-t-il alors utiliser 
	pour comparer ces deux fichiers \fich{Makefile} ?

\subsection*{Exercice 2}

La commande {\sc Unix} \fich{strings} est une commande qui s'applique à un 
fichier dont le nom est précisé en argument de la ligne de commande~: \\
	\unix{strings fichier}
 
Cette commande permet d'afficher à l'écran tous les caractères contenus dans 
\fich{fichier} dont les codes Ascii sont situés dans l'intervalle \mbox{$[32, 126]$}
(ce sont les caractères ``affichables'' du code Ascii).

{\bf Exemple~:}
si \fich{fichier} est constitué des caractères dont les codes Ascii 
sont  
\[
12, 32, 65, 69, 5, 4, 77, 89, 14, 35, 101, 102  
\]
alors {\tt strings fichier} affichera à l'écran les caractères dont les 
codes Ascii sont
\[
32, 65, 69, 77, 89, 35, 101, 102
\] 

Ecrivez en C la commande \fich{strings}.

\subsection*{Exercice 3}

Ecrivez un programme C \fich{min\_maj.c}, compilé sous le nom \fich{min\_maj},
qui affiche à l'écran ses arguments en convertissant toutes les lettres 
minuscules en lettres majuscules (les autres caractères étant inchangés).

Exemples~: \\
	\unix{min\_maj bOnNE nUIt LES petits ...} \\ 
affiche à l'écran : \\
	\unix{BONNE NUIT LES PETITS ...} \\
~\\
	\unix{min\_maj Vive Inf\_122 !} \\
affiche à l'écran : \\
	\unix{VIVE INF\_122 !}

\subsection*{Excercice 4}

Ecrivez un fichier de commandes de nom \fich{sauvegarde.sh} permettant de~: 
\begin{itemize}
\item 
	Créer un répertoire de nom \fich{Archives} dans votre répertoire 
	personnel (s'il n'y existe pas déjà)~;
\item
	Déplacez dans ce répertoire \fich{Archives} tous les fichiers exécutables 
	qui se trouvent dans le répertoire dont le nom est donné en argument. 
        Si le répertoire donné en argument n'existe pas, afficher un message d'erreur.    
\end{itemize}

\subsection*{Exercice 5}

L'administrateur d'une machine souhaite être prévenu lorsque le nombre
d'utilisateurs d'une machine dépasse une certaine valeur N.
Il souhaite donc écrire un programme qui lui affiche un message chaque fois
que cette valeur est dépassée.

Pour connaitre le nombre courant d'utilisateurs il dispose de la commande \fich{uptime}
(vue en TP)~:

\begin{verbatim}
{administrateur} 27 >uptime
  9:16am  up 99 day(s), 23:08,  2 users,  load average: 1.47, 1.60, 1.52
{administrateur} 28 >
\end{verbatim}

{\bf 1.} Ecrivez un fichier de commande \fich{uptime.sh} qui exécute 
	la commande \fich{uptime} toutes les 10 minutes

{\bf 2.} Ecrivez un programme C qui sera compilé sous le nom \fich{uptime\_vers\_users}.
	Ce programme prend en argument sur la ligne de commande un entier N. 
	Il lit en permanence au clavier le résultat de la commande \fich{uptime} et 
	affiche un message lorsque le nombre d'utilisateurs dépasse N. 

{\bf 3.} Quelle(s) commande(s) doit exécuter notre administrateur pour qu'un 
	message soit affiché chaque fois qu'il y a plus de 15 utilisateurs sur 
	la machine ?

\subsection*{Exercice 6}

La commande {\tt wc -l} affiche à l'écran le nombre de lignes du fichier
donné en arguments~:

\begin{verbatim}
{batman} 2 >wc -l debordechar.c 
      35 debordechar.c
{batman} 3 >
\end{verbatim}

On souhaite connaître le nombre total de lignes de tous les fichiers 
suffixés par {\tt .c} du répertoire courant.

{\bf 1.} Décrivez en quelques lignes le principe d'une solution possible.
	Cette solution peut combiner l'utilisation de commandes existantes 
	ainsi que des programmes C ou fichiers de commandes 
	(dont vous préciserez les fonctionnalités).

{\bf 2.} Donnez la réalisation complète de votre solution.

\end{document}



\subsection*{Exercice 6}

L'option {\tt -o} du compilateur {\tt gcc} permet d'indiquer le nom du 
programme exécutable que l'on souhaite générer. Cette option est ``dangereuse''
dans le sens où le compilateur ne fait aucune vérification sur le nom de ce 
programme exécutable. Ainsi la commande suivante efface silencieusement
le fichier source \fich{max2.c}~: \\
	\unix{gcc max2 -o max2.c}

Pour éviter ce genre de problème à un utilisateur débutant, on propose de 
lui écrire une nouvelle commande de nom \fich{compile}, qui accepte les mêmes 
arguments que \fich{gcc}, et qui~: 
\begin{itemize}
\item affiche un message d'avertissement si la commande proposée risque 
	d'effacer un fichier source ;
\item exécute normalement la commande standard \fich{gcc} dans le cas contraire.
\end{itemize}
~ \\

On suppose que cet utilisateur n'écrit et ne compile que des programmes constitués 
d'un seul fichier source. Il utilisera donc la commande \fich{compile} de deux 
manière possible : \\
	\unix{compile -o \mbox{{\em fichier1}} \mbox{{\em fichier2}}} \\
ou \\
	\unix{compile \mbox{{\em fichier1}} -o \mbox{{\em fichier2}}} 


{\bf 1.} 
Ecrivez la commande \fich{compile} sous la forme d'un fichier de commande.

{\bf 2.} 
Ecrivez la commande \fich{compile} sous la forme d'un programme C.

