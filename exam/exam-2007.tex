\documentclass[10pt]{article}
\usepackage[utf8]{inputenc}
\usepackage{esial}\CSH\1A
%\usepackage[correction]{esial}\CSH\1A
\newcommand{\cd}[1]{\medskip\noindent\file{\null\hspace{-1em}[#1] }}
\newcommand{\touche}[1]{\hbox{$<$#1$>$}}
\newcommand{\ctrl}[1]{\touche{ctrl-#1}}
\newcommand{\tab}{\touche{TAB}}
\sloppy
\usepackage{textcomp,amstext}

\newcommand{\unix}[1]{\hspace*{2cm}{\tt #1}}
\newcommand{\fich}[1]{{\bf \em #1}}

\newcommand{\BoxRep}{\ifcorrection{\boxtimes}{\Box}}

\title{Devoir surveillé du 26 mai 2007}
\begin{document}
\maketitle
\thispagestyle{empty}

\begin{quote}
  Tous documents interdits. Les exercices sont indépendants. La correction
  tiendra compte de la qualité de la rédaction et de la présentation. Barème
  approximatif.
\end{quote}


\begin{Exercice} 
  On souhaite écrire une fonction qui renvoie la position de la première
  occurence d'un caractère (paramètre \texttt{c}) dans une chaîne de caractères
  (paramètre \texttt{str}), ou -1 si le caractère n'est pas présent. Voici le
  prototype de la fonction:

  \run{int premier\_cara(char c, char *str);}
\Question \textbf{(2 pts)} Écrire cette fonction

\begin{Reponse}
  \begin{Verbatim}
int premier_cara(char c, char *str) {
  int i;
  for (i=0; str[i] != '\0; i++) 
    if (str[i] == c) 
      return i+1;
  return -1;
}
  \end{Verbatim}
\end{Reponse}
\Question \textbf{(1 pt)} Écrire une fonction \texttt{main} permettant de tester votre
travail. Exemple d'exécution:
\begin{Verbatim}
$ ./occurence o Bonjour
La première occurence de 'o' dans 'Bonjour' est à la position 2.
$  
\end{Verbatim}
\begin{Reponse}
  \begin{Verbatim}
#include <stdio.h>
int main(int argc,char *argv[]) {    
  printf("La première occurence de '%c' dans %s' est à la position %d.\n",
         argv[1][0], argv[2], premier_cara(argv[1][0], argv[2]));
  return 1;        
}
  \end{Verbatim}
\end{Reponse}
\end{Exercice}


\begin{Exercice} \textbf{(2 pts)}
  La fonction \texttt{strdup} de la bibliothèque standard permet de duppliquer
  une chaîne: elle réserve un emplacement mémoire suffisament grand pour
  contenir la chaîne, puis effectue la copie.

  \Question Réimplémentez \texttt{strdup} sans utiliser de fonction de la
  bibliothèque standard à part malloc.
\begin{Reponse}
  \begin{Verbatim}
char *chdup(char *str) {
  char *res;
  int i;
  for (i=0; str[i] != '\0'; i++);/* Calcule la longueur */
  
  res = malloc(i+1);             /* Alloue un espace suffisament grand (+1 pour \0) */

  for (i=0; str[i] != '\0'; i++) /* Fait la copie */
    res[i] = str[i];
  res[i] = '\0';
  return res;
}
  \end{Verbatim}
\end{Reponse}
\end{Exercice}

\begin{Exercice} \textbf{(2 pts)}
  Écrire un script shell \run{est\_arrive utilisateur} qui teste toutes les dix
  secondes si l'\texttt{utilisateur} en question s'est connecté, et s'arrête
  avec un message adéquat si c'est le cas. \texttt{utilisateur} est un nom de
  login UNIX. On rappelle que la commande \texttt{who} affiche les utilisateurs
  actuellement connectés sous la forme:
  \begin{Verbatim}
rouyerj2 pts/3        May 24 11:01
cario2   pts/46       May  2 08:59
costa15  pts/71       May  3 10:39
alexand2 pts/11       May 23 10:51 (arc.loria.fr)    
  \end{Verbatim}

\end{Exercice}
\begin{Reponse}
  \begin{Verbatim}
#! /bin/sh
while true ; # ou [ 1 -eq 1 ] par exemple
do
  if who | grep $1 ; then 
    echo $1 est arrivé!
    exit 0
  fi
  sleep 10
done    
  \end{Verbatim}
\end{Reponse}

\begin{Exercice} \textbf{(2 pts)}
  Écrire un script shell \run{danger dir} qui cherche tous les sous-répertoires
  de \texttt{dir} dans lesquels tout le monde peut écrire. 1 pt de bonus si votre
  solution mène une recherche récursive.
  \begin{Reponse}
    \begin{Verbatim}[label=Version de base]
#! /bin/sh

cd $1
for d in * ; do
  if [ -d $d ] ; then
      if ls -l $1 | grep $d | grep '^d.......w.' > /dev/null; then
            echo "Danger! Tout le monde peut écrire dans $1/$d"
      fi
  fi
done
\end{Verbatim}
\begin{Verbatim}[label=Version récursive]
#! /bin/sh

for d in $1/* ; do
  if [ -d $d ] ; then
      bn=`basename $d`
      if ls -lh $1 | grep $bn | grep '^d.......w.' > /dev/null; then
            echo "Danger! Tout le monde peut écrire dans $d"
      fi
      $0 $d
  fi
done
    \end{Verbatim}
  \end{Reponse}
\end{Exercice}

\begin{Exercice}\textbf{(2pts)}
%
Écrivez un fichier de commandes de nom \fich{sauvegarde.sh} permettant de~: 
\begin{itemize}
\item 
	Créer un répertoire de nom \fich{Archives} dans votre répertoire 
	personnel (s'il n'y existe pas déjà)~;
      \item Copier dans ce répertoire \fich{Archives} tous les fichiers
        exécutables qui se trouvent dans le répertoire dont le nom est donné en
        argument.  \\
        Si le répertoire donné en argument n'existe pas, afficher un message
        d'erreur.
\end{itemize}
\end{Exercice}
\begin{Reponse}\newpage
  \begin{Verbatim}
#! /bin/sh
if [ ! -e ~/Archives ] ; then
  mkdir ~/Archives
fi
if [ ! -e $1 ] ; then
  echo "$0: $1 n'existe pas"
  exit 1
fi
for fich in $1/* ; do
  if [ ! -d $fich ] ; then if [ -x $fich ] ; then
    cp $fich ~/Archives
  fi; fi
done
  \end{Verbatim}
\end{Reponse}%$
\begin{Exercice}\textbf{(3pts)} % Pour chacune des questions suivantes, vous répondrez en une
%  seule ligne de commande.
%   \begin{enumerate}
%   \item Envoyer la liste détaillée des fichiers du répertoire courant à toto@loria.fr
%   \item Idem, mais on ne souhaite pas afficher les informations concernant les
%     répertoires. 
%   \end{enumerate}
  On suppose que l'on dispose d'un fichier calepin.txt, contenant des noms et
  des numéros de téléphone rangés de la manière suivante :
  \begin{Verbatim}[label=calepin.txt,gobble=4]
    DUPONT Jean,05.61.75.18.47,21/08/1975,jean.dupont@free.fr
    MARTIN Yvonne,02.23.34.45.56,26/02/1977,yvonne.martin@cegetel.fr
    ...
  \end{Verbatim}

  \Question \textbf{(\textonehalf pt)} Écrire un script (ou une commande) qui
  effectue la recherche des personnes s'appelant DURAND et qui restitue leur
  addresse éléctronique.

  \begin{Reponse}
cat calepin.txt $|$ grep -i durand $|$ cut -f4 -d,
  \end{Reponse}

%  \Question \textbf{(\textonehalf pt)} Écrire un script (ou une commande) qui
%  compte les gens habitant dans le sud-est (numéro en 04)
%  \begin{Reponse}
%     cat calepin.txt | cut -f3 -d' ' | cut -f1 -d'.' | grep '05' | wc -l
%  \end{Reponse}
  \Question \textbf{(\textonehalf pt)} Écrire un script (ou une commande) qui
  compte les gens ayant 20 ans cette année.
  \begin{Reponse}
    cat calepin.txt $|$ cut -f3 -d, $|$ cut -f3 -d'/' $|$ grep -c 1987
  \end{Reponse}

%  \Question \textbf{(1pt)} Écrire un script (ou une commande) envoyant un
%  message éléctronique contenant ``Bonne Année'' à tout le monde.
  \Question \textbf{(2 pts)} Écrire un script (ou une commande) classant les
  personnes de la plus vieille à la plus jeune.
  \begin{Reponse}
    \begin{Verbatim}
cat calepin.txt | \
  sed 's|^\([^,]*,[^,]*\),\(..\)/\(..\)/\(....\)|\4/\3/\2,\1,\2/\3/\4|' | \
  sort -r | \
  sed 's|^[^,],||'
    \end{Verbatim}
    \begin{itemize}
    \item On écrit les lignes sous la forme:\\
      1975/08/21,DUPONT Jean,05.61.75.18.47,21/08/1975,jean.dupont@free.fr
    \item On les trie
    \item On retire le début qui nous a servi à trier
    \end{itemize}
  \end{Reponse}

%  \Question \textbf{(1pt)} Écrire un script (ou une commande) envoyant un
%  message éléctronique contenant ``Bon anniversaire'' à toutes les personnes
%  dont c'est l'anniversaire.
%
%  \noindent \textit{Indication}: \run{date +\%d/\%m} affiche la date du jour
%  sous la forme: jj/mm

  \noindent \textit{Indication}: \run{sort -r} trie les lignes passées entrée
  standard dans l'ordre décroissant.
\end{Exercice}

%\chead{}
\lhead{\LARGE NOM:}
\chead{\LARGE PRÉNOM:}
\newpage
\begin{Exercice}
  Voici un programme C. Il compile et s'exécute normalement sans erreur.

  \begin{minipage}{.35\linewidth}
  \begin{Verbatim}
#include <stdlib.h>
#include <stdio.h>
// Définition des fonctions
void f1 ( int a, int *b ) {
  a = *b;
}
void f2 ( int *b, int c ) {
  *b = c;
}
void f3 ( int *a, int c ) {
  f4 ( &a, c );
}
void f4 ( int **b, int a ) {
  **b = a;
}
  \end{Verbatim}
  \end{minipage}~~~~
  \begin{minipage}{.6\linewidth}
  \begin{Verbatim}
// Fonction principale
int main () {
  int x = 5, y = 7, z = 9;
  f1 ( x, &y );
  printf ( "x = %d, y = %d, z = %d\n", x, y, z );
  f2 ( &x, y );
  printf ( "x = %d, y = %d, z = %d\n", x, y, z );
  f3 ( &y, z );
  printf ( "x = %d, y = %d, z = %d\n", x, y, z );
  return 0;
}    
  \end{Verbatim}    
  \end{minipage}

\Question \textbf{(3 pts)}  Qu'affiche ce programme lors de son exécution?
\begin{Reponse}
\noindent x = 5, y = 7, z = 9\\
x = 7, y = 7, z = 9\\
x = 7, y = 9, z = 9

(-1/2 par erreur, mais l'exo est largement sur-noté. Il vaut pas 3 pts)
\end{Reponse}
\end{Exercice}

\begin{Exercice} \textbf{QCM}
Répondez sur la feuille fournie. Il peut y avoir plusieurs cases valides par
ligne; les réponses fausses seront pénalisées.

\Question \textbf{(2 pts)} Quel est le type de chacune des variables dans cet
extrait de programme?

\noindent\hspace{-.8em}\begin{minipage}{.28\linewidth}
\begin{Verbatim}
int *a,b;
char **c, *d[12];

typedef struct {
  char *marque;
  int nb_places;
  float consomation;
} voiture_t, *voiture; 

voiture_t e,f[12];
voiture   g,h[12];
\end{Verbatim}  
\end{minipage}~\begin{minipage}{.7\linewidth}
\begin{tabular}{|l|c|c|c|c|c|c|}\hline
  variable & entier & flottant & structure & pointeur & tableau & 
  $\left(\!\!\begin{array}{c}
    \text{écriture}\\
    \text{invalide}
  \end{array}\!\!\right)$ \\\hline
  a & $\Box$ & $\Box$ & $\Box$ & $\BoxRep$ & $\Box$ & $\Box$\\\hline
  b & $\BoxRep$ & $\Box$ & $\Box$ & $\Box$ & $\Box$ & $\Box$\\\hline
  c & $\Box$ & $\Box$ & $\Box$ & $\BoxRep$ & $\Box$ & $\Box$\\\hline
  d & $\Box$ & $\Box$ & $\Box$ & $\Box$ & $\BoxRep$ & $\Box$\\\hline
  e & $\Box$ & $\Box$ & $\BoxRep$ & $\Box$ & $\Box$ & $\Box$\\\hline
  e.nb\_places    & $\BoxRep$ & $\Box$ & $\Box$ & $\Box$ & $\Box$ & $\Box$\\\hline
  e-$>$nb\_places & $\Box$ & $\Box$ & $\Box$ & $\Box$ & $\Box$ & $\BoxRep$\\\hline
  f & $\Box$ & $\Box$ & $\Box$ & $\Box$ & $\BoxRep$ & $\Box$\\\hline
  f.nb\_places    & $\Box$ & $\Box$ & $\Box$ & $\Box$ & $\Box$ & $\BoxRep$\\\hline
  f-$>$nb\_places & $\Box$ & $\Box$ & $\Box$ & $\Box$ & $\Box$ & $\BoxRep$\\\hline
  f[0].nb\_places    & $\BoxRep$ & $\Box$ & $\Box$ & $\Box$ & $\Box$ & $\Box$\\\hline
  f[0]-$>$nb\_places & $\Box$ & $\Box$ & $\Box$ & $\Box$ & $\Box$ & $\BoxRep$\\\hline
  g & $\Box$ & $\Box$ & $\Box$ & $\BoxRep$ & $\Box$ & $\Box$\\\hline
  g.nb\_places    & $\Box$ & $\Box$ & $\Box$ & $\Box$ & $\Box$ & $\BoxRep$\\\hline
  g-$>$nb\_places & $\BoxRep$ & $\Box$ & $\Box$ & $\Box$ & $\Box$ & $\Box$\\\hline
  h & $\Box$ & $\Box$ & $\Box$ & $\Box$ & $\BoxRep$ & $\Box$\\\hline
  h.nb\_places    & $\Box$ & $\Box$ & $\Box$ & $\Box$ & $\Box$ & $\BoxRep$\\\hline
  h-$>$nb\_places & $\Box$ & $\Box$ & $\Box$ & $\Box$ & $\Box$ & $\BoxRep$\\\hline
  h[0].nb\_places    & $\Box$ & $\Box$ & $\Box$ & $\Box$ & $\Box$ & $\BoxRep$\\\hline
  h[0]-$>$nb\_places & $\BoxRep$ & $\Box$ & $\Box$ & $\Box$ & $\Box$ & $\Box$\\\hline
\end{tabular}
\end{minipage}
\begin{Reponse}
  \begin{itemize}
  \item +0.1 par réponse juste
  \item -0.1 par réponse fausse
  \item 0 si blanc ou une réponse juste et une réponse fausse sur la ligne
  \item arrondi au quart supérieur
  \end{itemize}
\end{Reponse}

\Question \textbf{(1 pt)}
\begin{itemize}
\item[$\bullet$] Si dans un fichier \textit{makefile} on trouve la ligne suivante :
  \run{toto: tutu}
  \begin{itemize}
  \item[$\BoxRep$] cela veut dire qu'il faut reconstruire \texttt{toto} chaque
    fois que \texttt{tutu} change
  \item[$\Box$] cela veut dire qu'il faut reconstruire \texttt{tutu} chaque
    fois que \texttt{toto} change
  \end{itemize}
\item[$\bullet$] Dans un fichier d'entêtes (.h), on peut trouver:
  \begin{itemize}
  \item[$\Box$] Des définitions de fonctions
  \item[$\BoxRep$] Des prototypes de fonctions
  \item[$\BoxRep$] Des définitions de types
  \end{itemize}
\item[$\bullet$] L'expression \run{if [ -x toto -a -d toto ]} teste si:
  \begin{itemize}
  \item[$\Box$] on a le droit d'exécuter et d'effacer \texttt{toto}
  \item[$\Box$] on a le droit d'exécuter \texttt{toto} et si sa taille est non nulle
  \item[$\Box$] \texttt{toto} est un répertoire dont on peut lire le contenu
  \item[$\BoxRep$] \texttt{toto} est un répertoire dans lequel on peut entrer
  \end{itemize}
\end{itemize}
\begin{Reponse}
  \begin{itemize}
  \item +0.25 par case cochée à juste titre
  \item -0.25 par case cochée à tord
  \item 0 si blanc
  \end{itemize}
\end{Reponse}

\end{Exercice}

      

% \begin{Exercice} \textbf{(3pts)}
%   %
%   L'administrateur d'une machine souhaite être prévenu lorsque le nombre
%   d'utilisateurs d'une machine dépasse une certaine valeur N.  Il souhaite donc
%   écrire un programme qui lui affiche un message chaque fois que cette valeur
%   est dépassée.

%   Pour connaître le nombre courant d'utilisateurs il dispose de la commande
%   \fich{uptime}:

% \begin{Verbatim}
% {administrateur} 27 >uptime
%   9:16am  up 99 day(s), 23:08,  2 users,  load average: 1.47, 1.60, 1.52
% {administrateur} 28 >
% \end{Verbatim}

% \Question \textbf{(\textonehalf pt)} Écrivez un fichier de commande
% \fich{uptime.sh} qui exécute la commande \fich{uptime} toutes les 10 minutes.

% \Question \textbf{(2pt)} Écrivez un programme C qui sera compilé sous le nom
% \fich{uptime\_vers\_users}.  Ce programme prend en argument sur la ligne de
% commande un entier N et une adresse email.  Il lit en permanence au clavier le
% résultat de la commande \fich{uptime} et envoie un couriel lorsque le nombre
% d'utilisateurs dépasse N. 

% \noindent \textit{Indication:} La fonction C \underline{\texttt{int
%     system(char *cmd)}} permet d'exécuter la commande \texttt{cmd} dans un
% shell. On peut utiliser le programme \texttt{mail} dans cette commande.

% \Question \textbf{(\textonehalf pt)} Quelle commande doit exécuter notre
% administrateur pour recevoir un couriel chaque fois qu'il y a plus de 15
% utilisateurs sur la machine ?
% \end{Exercice}

\end{document}



\subsection*{Exercice 6}

L'option {\tt -o} du compilateur {\tt gcc} permet d'indiquer le nom du 
programme exécutable que l'on souhaite générer. Cette option est ``dangereuse''
dans le sens où le compilateur ne fait aucune vérification sur le nom de ce 
programme exécutable. Ainsi la commande suivante efface silencieusement
le fichier source \fich{max2.c}~: \\
	\unix{gcc max2 -o max2.c}

Pour éviter ce genre de problème à un utilisateur débutant, on propose de 
lui écrire une nouvelle commande de nom \fich{compile}, qui accepte les mêmes 
arguments que \fich{gcc}, et qui~: 
\begin{itemize}
\item affiche un message d'avertissement si la commande proposée risque 
	d'effacer un fichier source ;
\item exécute normalement la commande standard \fich{gcc} dans le cas contraire.
\end{itemize}
~ \\

On suppose que cet utilisateur n'écrit et ne compile que des programmes constitués 
d'un seul fichier source. Il utilisera donc la commande \fich{compile} de deux 
manière possible : \\
	\unix{compile -o \mbox{{\em fichier1}} \mbox{{\em fichier2}}} \\
ou \\
	\unix{compile \mbox{{\em fichier1}} -o \mbox{{\em fichier2}}} 


{\bf 1.} 
Ecrivez la commande \fich{compile} sous la forme d'un fichier de commande.

{\bf 2.} 
Ecrivez la commande \fich{compile} sous la forme d'un programme C.


% LocalWords:  Makefile Laurel strings
%%% Local Variables:
%%% coding: utf-8
