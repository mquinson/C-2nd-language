\documentclass[10pt, sansserif,
               firstcolor=color1,
               secondcolor=color2,
               logo=logo-TN, 
               footband=bandeau-TN]{TelecomNancy}
\usepackage[utf8]{inputenc}
\author{AZE}
%\usepackage{esial}\CSH\1A
%\usepackage[correction]{esial}\CSH\1A
\newcommand{\cd}[1]{\medskip\noindent\file{\null\hspace{-1em}[#1] }}
\newcommand{\touche}[1]{\hbox{$<$#1$>$}}
\newcommand{\ctrl}[1]{\touche{ctrl-#1}}
\newcommand{\tab}{\touche{TAB}}
\sloppy
\usepackage{textcomp,amstext,wasysym}

\newcommand{\unix}[1]{\hspace*{2cm}{\tt #1}}
\newcommand{\fich}[1]{{\bf \em #1}}

%\newcommand{\BoxRep}{\Box}%Enoncé
\newcommand{\BoxRep}{X}%correction

\newenvironment{Reponse}{}{}
\let\Reponse=\comment  \let\endReponse=\endcomment
\newcommand{\run}[1]{\fbox{\texttt{#1}}}
\newcommand{\cmd}[1]{\texttt{#1}}
\newcommand{\file}[1]{{\bf \em #1}}
\newcommand{\result}[1]{\texttt{#1}}
\usepackage{fancyvrb}
\fvset{frame=single, numbersep=5pt}

\coursename{Première année}
\studentlevel{Langage C et Shell}
\doctitle{Consolidation de C, 11 juin 2015 (1h)~~~}
\definecolor{color1}{HTML}{6C2466}
\definecolor{color2}{HTML}{EF8A26}

\begin{document}
\globalinstructions[\vspace{-\baselineskip}~]{ %
  \hspace{-2em} Tous documents interdits à l'exception d'une page A4
  recto/verso, manuscrite de votre main. La correction tiendra compte
  de la qualité de l'argumentaire et de la présentation. Votre copie
  n'est pas un brouillon.}


\nextExercise[ROT13 (chaînes de caractères et fichiers -- 4pts)]{ Le
  \textit{ROT13 (rotate by 13 places) est un algorithme très simple de
    chiffrement de texte. Il s'agit d'un décalage de 13 caractères de
    chaque lettre du texte à chiffrer. Puisqu'il y a 26 lettres et que
    le décalage est de 13, il suffit d'appliquer deux fois de suite le
    chiffrement pour obtenir le texte en clair : la même procédure est
    utilisée pour le chiffrement et le déchiffrement. Pour cela on
    doit considérer l'alphabet comme circulaire, c'est-à-dire que
    Z+1=A. Les accentuées sont inchangées}.  \hfill(D'après Wikipédia).

  \medskip
  On souhaite écrire un programme appliquant ce chiffrement à une
  chaîne de caractères donnée.}

\bigskip\noindent
\resizebox{\linewidth}{!}{
  \begin{tabular}{ l|c c c c c c c c c c c c c c c c c c c c c c c c c c c }
    Caractère non-chiffré& A&B&C&D&E&F&G&H&I&J&K&L&M&N&O&P&Q&R&S&T&U&V&W&X&Y&Z 
    \\\hline
    Caractère chiffré    & N&O&P&Q&R&S&T&U&V&W&X&Y&Z&A&B&C&D&E&F&G&H&I&J&K&L&M
  \end{tabular}
}


\nextQuestion{Écrivez un programme complet prenant une chaîne de caractères à
  analyser en argument de la ligne de commande, de la manière suivante (2pts).}

\smallskip
\begin{Verbatim}
  $ ./rot13 "Bonjour les consolidés. Bon courage pour ce partiel."
  Obawbhe yrf pbafbyvqéf. Oba pbhentr cbhe pr cnegvry.
  $ ./rot13 `./rot13 "Bonjour les consolidés. Bon courage pour ce partiel."`
  Bonjour les consolidés. Bon courage pour ce partiel.
\end{Verbatim}
%$

\nextQuestion{Modifiez votre programme afin qu'il lise le texte à analyser
  depuis un fichier dont le nom est passé en argument de la ligne de commandes
  (2pt).}

%%%%%%%%%%%%%%%%%%%%%%%%%%%

\vspace{-.5\baselineskip}

\nextExercise[Gestion de BDthèque (programmation structurée -- 6pts)]{

  \vspace{-\baselineskip}
  Afin de ne plus égarer les bandes dessinées que je prête autour de
  moi, je vous demande de m'aider à réaliser un petit programme de
  gestion de ma collection. Chaque ouvrage est défini par son genre
  (manga, SF, medfan, historique, qui sont les seuls genres de ma
  collection), son titre et sa date d'achat. Chaque ouvrage peut de
  plus être disponible ou bien prété, auquel cas je dois enregistrer
  le nom de la personne à qui j'ai prêté mon livre, et la date.}

\nextQuestion{Écrivez une structure de données \texttt{bd\_t} contenant
  toutes les informations relatives à une BD donnée. Vous suivrez
  l'approche plutôt orientée objet vue en cours. (1pt)}
 
\nextQuestion{Écrivez la fonction \texttt{bd\_new()}, prenant en
  arguments les informations initiales, et retournant une référence à
  un «objet», c'est-à-dire un pointeur sur une nouvelle structure
  \texttt{bd\_t} (2pt).}

\nextQuestion{Écrivez la fonction \texttt{bd\_free()} prenant une
  référence à une BD pour libérer la mémoire occupée (1pt).\hspace{-2em}}
\nextQuestion{Écrivez une fonction \texttt{bd\_est\_dispo()}
  retournant si oui ou non la BD en paramètre est chez moi.}
\nextQuestion{Écrivez une fonction \texttt{bd\_pret(bd\_t BD, char
    *qui, char *quand)} qui s'assure que la BD en paramètre est
  marquée comme prêtée à la personne indiquée (1pt).}

% \nextQuestion{Écrivez la déclaration d'un tableau nommé \texttt{voitures},
%   permettant de stocker les informations relatives à toutes les voitures du
%   loueur. Une variable \texttt{nb\_voitures} stockera la taille de ce tableau
%   (\textonehalf pt).}

% \nextQuestion{Écrivez une fonction \texttt{voiture\_achete()} créant une
%   nouvelle voiture grâce à \texttt{voiture\_new()} ci-dessus et qui l'ajoute au
%   tableau. Comment être sûr que le tableau est assez grand ? (2pt)}

% \nextQuestion{Écrivez une fonction qui étant donné une catégorie de voitures,
%   retourne un pointeur vers la première voiture de cette catégorie qui ne soit
%   pas encore louée, ou NULL si aucune n'est disponible (1pt).}

%%%%%%%%%%%%%%%%%%%%%%%%%%%%
\newpage\fancyhead{}\fancyfoot{}
\setlength{\headheight}{25mm}
\null\vspace{-6\baselineskip}

%%%%%%%%%%%%%%%%%%%%%%%%
\lhead{\LARGE NOM:}
\chead{\LARGE PRÉNOM:}

\vspace{2.5\baselineskip} \nextExercise[QCM (10pts)]{Répondez sur
  cette feuille. Une seule case juste par question. Les réponses
  fausses seront pénalisées.}

\smallskip\noindent
  \begin{minipage}[t]{.3\linewidth}
  \begin{Verbatim}[label=Début du programme,numbers=left]
#include <stdlib.h>
#include <stdio.h>
void f1 (int a, int *b){
  a = *b;
}
void f2 (int *b, int c){
  *b = c;
}
void f3 (int *a, int c){
  f4 ( &a, c );
}
void f4 (int **b, int a){
  **b = a;
}
  \end{Verbatim}
  \end{minipage}~~~~~
  \begin{minipage}[t]{.6\linewidth}
  \begin{Verbatim}[label=Fin du programme,numbers=right,firstnumber=last]
int main () {
  int x = 5, y = 7, z = 9;
  f1 ( x, &y );
  printf ( "x = %d, y = %d, z = %d\n", x, y, z );
  f2 ( &x, y );
  printf ( "x = %d, y = %d, z = %d\n", x, y, z );
  f3 ( &y, z );
  printf ( "x = %d, y = %d, z = %d\n", x, y, z );
}    
  \end{Verbatim}    
  \vspace{-1.7\baselineskip}
  \nextQuestion{
  Qu'affiche la ligne 18? x = \framebox{\phantom{\Huge 12}}
  y = \framebox{\phantom{\Huge 12}}
  z = \framebox{\phantom{\Huge 12}}}

  \nextQuestion{Qu'affiche la ligne 20? x = \framebox{\phantom{\Huge 12}}
  y = \framebox{\phantom{\Huge 12}}
  z = \framebox{\phantom{\Huge 12}}}

  \nextQuestion{Qu'affiche la ligne 22? x = \framebox{\phantom{\Huge 12}}
  y = \framebox{\phantom{\Huge 12}}
  z = \framebox{\phantom{\Huge 12}}}
  \end{minipage}

%%%%%%%%%%%%%%%%%%%%%%%%%%

\medskip
\noindent\hspace{-.8em}\begin{minipage}{.23\linewidth}
\nextQuestion{Quel est le type de chacune des variables dans cet
extrait de programme? (4pts)}

\begin{Verbatim}
typedef struct {
  char *marque;
  int nb_places;
}  voiture_t, 
  *voiture; 

voiture_t a,b[12];
voiture   c,d[12];
\end{Verbatim}  
\end{minipage}~\begin{minipage}{.7\linewidth}
\begin{tabular}{|l|c|c|c|c|c|c|}\hline
  variable & entier & flottant & structure & pointeur & tableau & \textit{(invalide)}
  \\\hline
  a & $\Box$ & $\Box$ & $\BoxRep$ & $\Box$ & $\Box$ & $\Box$\\\hline
  a.nb\_places    & $\BoxRep$ & $\Box$ & $\Box$ & $\Box$ & $\Box$ & $\Box$\\\hline
  a-$>$nb\_places & $\Box$ & $\Box$ & $\Box$ & $\Box$ & $\Box$ & $\BoxRep$\\\hline
  b & $\Box$ & $\Box$ & $\Box$ & $\Box$ & $\BoxRep$ & $\Box$\\\hline
  b.nb\_places    & $\Box$ & $\Box$ & $\Box$ & $\Box$ & $\Box$ & $\BoxRep$\\\hline
  b-$>$nb\_places & $\Box$ & $\Box$ & $\Box$ & $\Box$ & $\Box$ & $\BoxRep$\\\hline
  b[0].nb\_places    & $\BoxRep$ & $\Box$ & $\Box$ & $\Box$ & $\Box$ & $\Box$\\\hline
  b[0]-$>$nb\_places & $\Box$ & $\Box$ & $\Box$ & $\Box$ & $\Box$ & $\BoxRep$\\\hline
  c & $\Box$ & $\Box$ & $\Box$ & $\BoxRep$ & $\Box$ & $\Box$\\\hline
  c.nb\_places    & $\Box$ & $\Box$ & $\Box$ & $\Box$ & $\Box$ & $\BoxRep$\\\hline
  c-$>$nb\_places & $\BoxRep$ & $\Box$ & $\Box$ & $\Box$ & $\Box$ & $\Box$\\\hline
  d & $\Box$ & $\Box$ & $\Box$ & $\Box$ & $\BoxRep$ & $\Box$\\\hline
  d.nb\_places    & $\Box$ & $\Box$ & $\Box$ & $\Box$ & $\Box$ & $\BoxRep$\\\hline
  d-$>$nb\_places & $\Box$ & $\Box$ & $\Box$ & $\Box$ & $\Box$ & $\BoxRep$\\\hline
  d[0].nb\_places    & $\Box$ & $\Box$ & $\Box$ & $\Box$ & $\Box$ & $\BoxRep$\\\hline
  d[0]-$>$nb\_places & $\BoxRep$ & $\Box$ & $\Box$ & $\Box$ & $\Box$ & $\Box$\\\hline
\end{tabular}
\end{minipage}
% \begin{Reponse}
%   \begin{itemize}
%   \item +0.1 par réponse juste
%   \item -0.1 par réponse fausse
%   \item 0 si blanc ou une réponse juste et une réponse fausse sur la ligne
%   \item arrondi au quart supérieur
%   \end{itemize}
% \end{Reponse}

\nextQuestion{Si dans un fichier \textit{makefile} on trouve la ligne
  suivante : \run{toto: tutu} cela veut dire (1pt):}

$\BoxRep$ il faut reconstruire \texttt{toto} quand \texttt{tutu} change
~~~~~~~~
$\Box$ il faut reconstruire \texttt{tutu}
  quand \texttt{toto} change

\nextQuestion{Que ne devrait-on PAS trouver dans les fichiers
  d'entêtes (.h) d'un programme bien écrit (1pt):}\smallskip

  $\BoxRep$    Des définitions de fonctions~~~~~~
  $\Box$ Des prototypes de fonctions~~~~~~
  $\Box$ Des définitions de types

\medskip\noindent
\begin{minipage}{.65\linewidth}
  \nextQuestion{Que fait le code ci-contre? (1pt)}  
  ~~~~~~$\Box$ Elle affecte à p la valeur -5~~~ 
        $\Box$ Elle affecte à p la valeur -1

  ~~~~~~$\Box$ L'adresse de p vaut -3~~~~~~~~~
        $\Box$ L'adresse de p vaut -1

  ~~~~~~$\BoxRep$ Elle affecte à troisième case entière de p la valeur -3

%  ~~~~~~$\Box$    Elle affecte à troisième case entière de p la valeur -5
\end{minipage}
\begin{minipage}{.35\linewidth}
  \begin{Verbatim}
int* p=malloc(3*sizeof(int));
*(p + 2) = -3;
  \end{Verbatim}
\end{minipage}
% \begin{Reponse}
%   \begin{itemize}
%   \item +0.25 par case cochée à juste titre
%   \item -0.25 par case cochée à tord
%   \item 0 si blanc
%   \end{itemize}
% \end{Reponse}

%\end{Exercice}

      
\end{document}



% LocalWords:  Makefile Laurel strings
%%% Local Variables:
%%% coding: utf-8
