\documentclass[10pt]{article}\usepackage[enonce]{exemptty}

%\usepackage{exercices}[correction]
%\usepackage[hidelinks]{hyperref}
%\usepackage{../exercices}
%\usepackage{color}
\usepackage{booktabs}

\newcommand{\I}{\hspace{1.5em}}

\title{Allocation de tableaux en C}
\TDnumber{0}
\TDmodule{}%Module ArcSys}

% Pour l'année prochaine :
% - Améliorer le makefile pour un meilleur affichage de succès/failure
% - Supprimer des headers le fait de renvoyer NULL pour les allocations invalide (on appelle abort() jamais autre chose
% !!)

\begin{document}

\maketitle

\noindent\framebox{\begin{minipage}{.95\linewidth}
    \noindent\textbf{\large Objectifs pédagogiques:}
    \begin{itemize}
    \item[$\bullet$] Allouer/libérer de la mémoire avec \lstinline$malloc$/\lstinline$free$;
    \item[$\bullet$] Manipuler des pointeurs/tableaux;
    \item[$\bullet$] Créer de nouveaux types de données;
    \item[$\bullet$] Utiliser en pratique les outils gdb et valgrind.
    \end{itemize}
\end{minipage}}\medskip

L'objectif de ce TP est de manipuler les fonctions \lstinline$malloc$
et \lstinline$free$, et de vous entrainer à debugger un programme
s'arrêtant sur SEGFAULT. L'excuse du jour sera la création d'une
bibliothèque permettant de créer et manipuler des vecteurs et des
matrices.  \bigskip

\PseudoExo{Principe du TP.} 
Veuillez télécharger l'archive :\\
\url{http://people.irisa.fr/Martin.Quinson/Teaching/ArcSys/99-allocation-tableaux.tar.gz}

Vous trouverez dans cette archive plusieurs programmes à compléter. Vous pouvez
compiler et tester tous les exercices de la séance avec la commande \run{make
  test}

Chaque test correspond à un programme (le nom du programme s'affiche avant
ECHEC ou SUCCES) que vous pouvez exécuter indépendemment pour déterminer vos
erreurs. Le test \texttt{$<$toto$>$} est passé si votre programme \texttt{toto}
affiche exactement la même chose que ce qui se trouve dans le fichier
\texttt{toto.result}. N'hésitez pas à consulter le contenu des fichiers fournis
pour comprendre le fonctionnement de l'ensemble. Si votre programme vous semble
fonctionner, mais que la suite de tests vous indique le contraire, comparez le
contenu du fichier \texttt{toto.output} produit par votre programme au fichier
\texttt{toto.result}, qui est la sortie attendue par la suite de tests. Pour
cela, vous pouvez utiliser la commande

\noindent\run{diff -u toto.result toto.output}
qui affiche les différences ligne par ligne. Il est bien entendu possible de
modifier \texttt{toto.result} pour faire en sorte que les tests ne détectent
plus le problème, mais ce n'est pas l'objectif ;)

Pour comprendre vos segfaults, utilisez valgrind en exécutant:
\run{valgrind --leak-check=full ./toto}

\begin{Exercice} \textbf{Vecteurs.}
  Nous souhaitons réaliser les fonctions nécessaires à la gestion d'un type
  vecteur.

  \Question Complétez \file{vector.c} qui contient toutes les
  fonctions de gestion de vecteur que nous voulons implémenter. Le
  fichier \file{vector.h} contient toutes les informations
  nécessaires : description des fonctions et déclaration du type
  vecteur. Une fois toutes les fonctions complétées, tapez \run{make
    t11-vector} pour compiler, \run{./t11-vector} pour exécuter le
  programme de test et make test pour tester si le résultat du
  programme de test est correct (s'il affiche la même chose que ce qui
  se trouve dans \file{t11-vector.result}).

  \Question Ajoutez un test dans la fonction d'accès, et renvoyez NULL
  si l'indice demandé est en dehors des bornes de l'objet
  concerné. Testez votre modification avec le programme
  \run{./t12-vector-bound}

  \Question On a supposé à la question précédente qu'accéder à une
  case inexistante est une erreur. Parfois, on préfère faire en sorte
  que le vecteur grandisse automatiquement pour créer à la demande les
  cases inexistantes. Comme il est difficile de prédire le
  comportement souhaité par l'utilisateur de la bibliothèque, on va
  configurer le comportement avec le champ \texttt{dynamique} de la
  structure vecteur. S'il vaut FAUX (\texttt{dynamique == 0}), on
  utilisera le comportement de la question précédente: un accès à une
  case inexistante renvoi NULL. Si le champ vaut VRAI
  (\texttt{dynamique != 0}), il faut agrandir le tableau (avec
  \texttt{realloc()}) quand l'indice dépasse la borne.

  Testez votre modification avec le programme \run{./t13-vector-realloc}.
\end{Exercice}

\begin{Exercice} \textbf{Matrices.}  On veut implémenter un type
  \texttt{matrix\_t}, représentant les matrices ainsi:
  
  \smallskip\noindent
  \begin{minipage}{.5\linewidth}%

    ~~Pour allouer une telle structure, on alloue d'abord un tableau de
    pointeurs, dont la taille correspond au nombre de lignes
    souhaité. Ensuite, chacune des cases de ce premier tableau est un
    pointeur vers un autre tableau alloué séparément pour représenter
    l'une des lignes. Le code correspondant est donné page suivante.

    \smallskip
    ~~On accède ensuite aux cases du tableaux comme on le ferait pour un
    tableau alloué statiquement: avec l'opérateur \verb~[ ]~ pour
    chacune des dimensions:

\begin{Verbatim}[label=Initialisation des éléments,numbers=right]    
for (int i = 0; i < 10; i++)
    for (int j = 0; j < 10; j++)
        tableau[i][j] = 0;
\end{Verbatim}

\end{minipage}\hfill\begin{minipage}{.48\linewidth}
  \includegraphics[width=\linewidth]{tableau.pdf}
\end{minipage}
  
\begin{Verbatim}[label=Allocation d'un tableau de 10*10 doubles]
double** tableau;

tableau = malloc(sizeof(double*) * 10);
for (int i = 0; i < 10; i++) {
    tableau[i] = malloc(sizeof(double) * 10);
}
\end{Verbatim}

\begin{Verbatim}[label=Idem avec gestion des erreurs]
double** tableau;

tableau = malloc(sizeof(double *) * 10);
if (tableau == NULL) { // malloc signale une erreur
    abort(); // tue le programme en signalant un bug interne
}
for (int i = 0; i < 10; i++) {
    tableau[i] = malloc(sizeof(double) * 10);
    if (tableau[i] == NULL) {
        abort();
    }
}
\end{Verbatim}


  \Question Complétez le fichier \file{matrix.c} qui contient toutes les
  fonctions de gestion de matrice que nous voulons implémenter. Le fichier
  \file{matrix.h} contient toutes les informations nécessaires : description
  des fonctions et déclaration du type \texttt{matrix\_t}. Une fois les fonctions
  complétées, tapez \run{make t21-matrix} pour compiler,
  \run{./make t21-matrix} pour exécuter le programme de test et \run{make
    test} pour tester si le résultat du programme de test est correct.

  \Question Modifiez votre fonction d'accès pour renvoyer NULL si
  l'on accède à une case qui n'existe pas, et testez votre travail
  avec \run{./t22-matrix-bounds}.

  \Question Modifiez maintenant votre fonction d'accès pour qu'elle
  augmente la taille de la matrice lorsqu'on accède à une case qui
  n'existe pas encore, à condition que la matrice soit
  dynamique. Testez votre travail avec \run{./t23-matrix-realloc}
  
\end{Exercice}


\medskip
\begin{Exercice} \textbf{Matrices linéaires (optionnel)} 

  En pratique, les matrices sont très rarement représentées en mémoire
  comme nous avons fait à l'exercice précédent. Au lieu de cela, on
  utilise un seul gros tableau unidimensionel. Les lignes de la
  matrice sont alors stockées les unes après les autres, de façon
  contiguë. La fonction d'accès traduit un couple d'indices de la
  matrice en un unique indice dans le tableau.
%
  On économise ainsi une indirection, ainsi que le coût des structures
  de gestion mémoire associées à chaque \texttt{malloc} par le
  système.  Les tableaux multidimensionels statiques sont
  automatiquement stockés sous cette forme en mémoire.

  \centerline{\includegraphics[width=\linewidth]{tableau-lineaire.pdf}}

  \Question Implémentez le type \texttt{matlin\_t} (pour
  \textit{mat}rice \textit{lin}éaire) dans le fichier
  \file{matlin.c}. Comme d'habitude, la documentation se trouve dans
  \file{matlin.h} et vous pouvez tester votre travail avec
  \run{./t31-matlin}.

  \Question Étendez votre fonction d'accès afin de retourner NULL
  quand on accède à une case inexistante, et testez votre travail avec
  \run{./t32-matlin-bounds}.

  \Question Étendez à nouveau votre fonction d'accès afin d'agrandir
  la matrice lorsqu'on accède à une case inexistante d'une matrice
  dynamique, puis testez votre travail avec \run{./t33-matlin-realloc}.
\end{Exercice}


\begin{Exercice}\textbf{Opérations mémoire (optionnel)}
Complétez le fichier \file{memory.c} afin d'implémenter les fonctions de manipulation mémoire suivantes:
\begin{itemize}
\item \texttt{my\_memcpy}: copie d'une zone en mémoire de la même manière que
  \texttt{memcpy} (cf. man);
\item \texttt{my\_memmove}: copie d'une zone en mémoire avec
  recouvrement possible, c'est-à-dire que la zone à écrire peut
  chevaucher partiellement la zone à lire (cf. \texttt{man
    memmove}). Il est parfois nécessaire de commencer par recopier les
  octets de la fin;
\item \texttt{is\_little\_endian}: renvoie vrai si l'architecture
  cible utilise la convention \textit{little endian};
\item \texttt{reverse\_endianess}: renvoie la valeur passée en
  argument avec ses octets inversé.
\end{itemize}
Vérifiez votre travail avec \run{./t41-memory-cpy} \run{./t42-memory-move} et
\run{./t43-memory-endianess}.
\end{Exercice}
\end{document}
