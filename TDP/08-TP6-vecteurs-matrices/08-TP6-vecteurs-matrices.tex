\documentclass[10pt]{article}\usepackage[enonce]{exemptty}

%\usepackage{exercices}[correction]
%\usepackage[hidelinks]{hyperref}
%\usepackage{../exercices}
%\usepackage{color}
\usepackage{booktabs}

\newcommand{\I}{\hspace{1.5em}}

\title{Allocation de tableaux en C}
\TDnumber{0}
\TDmodule{}%Module ArcSys}

% Pour l'année prochaine :
% - Améliorer le makefile pour un meilleur affichage de succès/failure
% - Supprimer des headers le fait de renvoyer NULL pour les allocations invalide (on appelle abort() jamais autre chose
% !!)

\begin{document}

\maketitle

\noindent\framebox{\begin{minipage}{.95\linewidth}
        \noindent\textbf{\large Objectifs pédagogiques:}
        \begin{itemize}
		\item[$\bullet$] Allouer/libérer de la mémoire avec \lstinline$malloc$/\lstinline$free$;
		\item[$\bullet$] Manipuler des pointeurs/tableaux;
		\item[$\bullet$] Créer de nouveaux types de données;
		\item[$\bullet$] Utiliser en pratique les outils gdb et valgrind.
        \end{itemize}
\end{minipage}}\medskip

%L'objectif de ce TP est d'appliquer les méthodes de debug vues au TP~5
%pour mettre au point des programmes manipulant la mémoire. L'excuse du jour
%sera la création d'une bibliothèque permettant de créer et manipuler des
%vecteurs et des matrices.
%\bigskip

\PseudoExo{Introduction.} En C, l'allocation dynamique d'un tableau
multidimensionnel se fait
en deux étapes :\\
\begin{minipage}{.5\linewidth}
  \begin{itemize}
  \item allocation d'un tableau de pointeurs dont la taille correspond au
    nombre de lignes souhaité
  \item allocation de chacune des lignes en utilisant le tableau de pointeurs
    précédent pour stocker leur adresse
  \end{itemize}

  Ensuite, l'accès au tableau se fait comme d'habitude en C, en utilisant
  l'opérateur \verb~[]~ pour préciser l'indice dans chacune des dimensions. La seule
  différence avec un tableau alloué de manière statique est le type des objets
  manipulés : dans le cas dynamique ce sera un tableau de pointeurs, dans le
  cas statique ce sera un tableau de tableaux (les tableaux sont réellement
  stockés les uns après les autres en mémoire).

\end{minipage}\hfill\begin{minipage}{.48\linewidth}
  \includegraphics[width=\linewidth]{tableau.pdf}
\end{minipage}


\begin{Verbatim}[label=Exemple: allocation d'un tableau de 10*10 doubles (sans aucune vérification)]
double **tableau;

tableau = malloc(sizeof(double *) * 10);
for (i = 0; i < 10; i++) {
    tableau[i] = malloc(sizeof(double) * 10);
}
\end{Verbatim}

\noindent\begin{minipage}{.64\linewidth}
\begin{Verbatim}[label=Idem avec gestion des erreurs]
double **tableau;

tableau = malloc(sizeof(double *) * 10);
if (!tableau) {
    abort();
}
for (i = 0; i < 10; i++) {
    tableau[i] = malloc(sizeof(double) * 10);
    if (!tableau[i]) {
        abort();
    }
}
\end{Verbatim}
\end{minipage}
\hfill
\begin{minipage}{.35\linewidth}
\begin{Verbatim}[label=Initialisation des éléments,numbers=right]    
for (i = 0; i < 10; i++)
    for (j = 0; j < 10; j++)
        tableau[i][j] = 0;
\end{Verbatim}

(vous pouvez aussi utiliser la fonction \verb~calloc~ pour allouer et initialiser à zéro les lignes)
\end{minipage}

\medskip

Veuillez télécharger l'archive :\\
\url{http://people.irisa.fr/Martin.Quinson/Teaching/ArcSys/99-allocation-tableaux.tar.gz}

Vous trouverez dans cette archive plusieurs programmes à compléter. Vous pouvez
compiler et tester tous les exercices de la séance avec la commande \run{make
  test}

Chaque test correspond à un programme (le nom du programme s'affiche avant
ECHEC ou SUCCES) que vous pouvez exécuter indépendemment pour déterminer vos
erreurs. Le test \texttt{$<$toto$>$} est passé si votre programme \texttt{toto}
affiche exactement la même chose que ce qui se trouve dans le fichier
\texttt{toto.result}. N'hésitez pas à consulter le contenu des fichiers fournis
pour comprendre le fonctionnement de l'ensemble. Si votre programme vous semble
fonctionner, mais que la suite de tests vous indique le contraire, comparez le
contenu du fichier \texttt{toto.output} produit par votre programme au fichier
\texttt{toto.result}, qui est la sortie attendue par la suite de tests. Pour
cela, vous pouvez utiliser la commande

\noindent\run{diff -u toto.result toto.output}
qui affiche les différences ligne par ligne. Il est bien entendu possible de
modifier \texttt{toto.result} pour faire en sorte que les tests ne détectent
plus le problème, mais ce n'est pas l'objectif ;)

Si vous souhaitez utiliser valgrind (et c'est conseillé), vous pouvez exécuter :
\begin{verbatim}
valgrind --leak-check=full ./mon_programme
\end{verbatim}
\clearpage

\begin{Exercice} \textbf{Vecteurs}

  Nous souhaitons réaliser les fonctions nécessaires à la gestion d'un type
  vecteur. Complétez \file{vecteur.c} qui contient toutes les fonctions de
  gestion de vecteur que nous voulons implémenter. Le fichier \file{vecteur.h}
  contient toutes les informations nécessaires : description des fonctions et
  déclaration du type vecteur. Une fois les fonctions complétées, tapez
  \run{make vecteur\_testbase} pour compiler, \run{./vecteur\_testbase} pour
  exécuter le programme de test et make test pour tester si le résultat du
  programme de test est correct (s'il affiche la même chose que ce qui se
  trouve dans \file{vecteur\_testbase.result}).
\end{Exercice}

\begin{Exercice} \textbf{Matrices}

  Nous souhaitons réaliser les fonctions nécessaires à la gestion d'un type
  matrice, où les matrices sont représentées par un tableau bidimensionel
  dynamique. Complétez le fichier \file{matrice.c} qui contient toutes les
  fonctions de gestion de matrice que nous voulons implémenter. Le fichier
  \file{matrice.h} contient toutes les informations nécessaires : description
  des fonctions et déclaration du type matrice. Une fois les fonctions
  complétées, tapez \run{make matrice\_testbase} pour compiler,
  \run{./matrice\_testbase} pour exécuter le programme de test et \run{make
    test} pour tester si le résultat du programme de test est correct.
\end{Exercice}

\begin{Exercice} \textbf{Matrices linéaires (optionnel)} 

  Nous souhaitons réaliser les fonctions nécessaires à la gestion d'un type
  matrice, où les matrices sont représentées par un seul tableau
  unidimensionel. Dans ce cas, les lignes de la matrice sont stockées les unes
  après les autres et la fonction d'accès va devoir faire la traduction d'un
  couple d'indices vers un unique indice dans le tableau (l'avantage est que
  l'on économise une indirection ainsi que le coût des structures de gestion
  mémoire associées à chaque \texttt{malloc} par le système). La figure
  suivante décrit l'organisation de nos matrices en mémoire :

  \centerline{\includegraphics[width=\linewidth]{tableau-lineaire.pdf}}

  Complétez le fichier \file{matrice\_lineaire.c} qui contient toutes les
  fonctions de gestion de matrice que nous voulons implémenter. Le fichier
  \file{matrice\_lineaire.h} contient toutes les informations nécessaires :
  description des fonctions et déclaration du type matrice. Une fois les
  fonctions complétées, tapez
  
  \noindent\run{make matrice\_lineaire\_testbase} pour
  compiler, \run{./matrice\_lineaire\_testbase} pour exécuter le programme de
  test et \run{make test} pour tester si le résultat du programme de test est
  correct.
\end{Exercice}

\begin{Exercice} \textbf{Vérifier les bornes (optionnel)}

  Reprenez les trois exercices précédents en ajoutant un test dans la fonction
  d'accès permettant de renvoyer NULL si les indices demandés sont en dehors
  des bornes de l'objet concerné.
\end{Exercice}

\begin{Exercice} \textbf{Réallocation dynamique (optionnel)}

  Reprenez les trois premiers exercices en ajoutant une réallocation dynamique
  des objets dans la fonction d'accès. Le comportement de cette fonction doit
  alors être le suivant :
  \begin{itemize}
  \item si les indices sont dans les bornes de l'objet retourner le pointeur
    d'accès au bon élément;
  \item si un des indices dépasse les bornes de l'objet tenter de réallouer
    plus de mémoire et renvoyer le pointeur d'accès au bon élément si la
    réallocation réussit et \verb~abort()~ sinon.
  \end{itemize}
  
  Pour la réallocation, on pourra utiliser au choix un \texttt{malloc} d'un
  bloc plus gros suivi d'une copie ou bien l'appel système \texttt{realloc}
  dont on obtient la description avec la commande \run{man 3 realloc}.

  Complétez \file{vecteur\_dynamique.c}, \file{matrice\_dynamique.c} et
  \file{matrice\_lineaire\_dynamique.c}.
\end{Exercice}

\begin{Exercice}\textbf{Opérations mémoire (optionnel)}
Implémentez les fonctions de manipulation mémoire suivantes :
\begin{itemize}
\item \texttt{my\_memcpy} : copie d'une zone en mémoire de la même manière que
  \texttt{memcpy} (cf. man);
\item \texttt{my\_memmove} : copie d'une zone en mémoire avec recouvrement
  possible.  (cf. \texttt{man memmove});
\item \texttt{is\_little\_endian} : renvoie vrai si l'architecture cible
  utilise la convention \textit{little endian} pour la représentation des entiers en
  mémoire;
\item \texttt{reverse\_endianess} : renvoie la valeur passée en argument avec
  ses octets inversé.
\end{itemize}
  Le fichier à compléter est \file{memory\_operations.c}. 
\end{Exercice}
\end{document}
