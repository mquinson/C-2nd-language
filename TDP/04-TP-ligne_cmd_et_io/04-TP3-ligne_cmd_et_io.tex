\documentclass[10pt]{article}\usepackage[correction,nu]{esial}
\CSH\unA\sloppy

\usepackage[utf8]{inputenc}
\hypersetup{colorlinks=false,pdfborder={0 0 0}}
\begin{document}
\title{TP3: Arguments en ligne de commande et E/S~~~~~~~}
\maketitle

\Exercice\textbf{Échauffement.}

\Question Écrivez un programme C acceptant un nombre variable d'arguments sur
la ligne de commande et affiche pour chacun d'eux le nombre de caractères
correspondant. Ainsi, si ce programme est compilé sous le nom
\file{longueur\_arg}, l'exécution de la commande\\
\run{longueur\_arg 0 bonjour 2.56 adieu} produira la sortie~:
\begin{verbatim}
   l'argument no 0 contient 12 caractere(s)
   l'argument no 1 contient 1 caractere(s)
   l'argument no 2 contient 7 caractere(s)
   l'argument no 3 contient 4 caractere(s)
   l'argument no 4 contient 5 caractere(s)
\end{verbatim}

\noindent\underline{Indication:} On peut retrouver la longueur d'une chaîne de
caractères avec la fonction \texttt{strlen()}, utilisable après avoir inclu le
fichier d'entête \texttt{<string.h>}.

\Exercice\textbf{Les arguments de la ligne de commande sont des chaînes
de caractères.}

\Question Écrivez un programme duppliquant un fichier \textit{source} sous le
nom \textit{destination} (que vous demanderez à l'utilisateur). Pour la copie,
vous utiliserez les fonctions \texttt{fprintf} et \texttt{fscanf} avec la
chaine de formatage \texttt{"\%c"}.

\begin{Reponse}
  La première chose à laquelle il faut les amener est un
  \begin{Verbatim}
while (!feof(IN)) {
  char c;
  fscanf(IN, "%c", &c);
  fprintf(IN, "%c", c);    
}
  \end{Verbatim}
  C'est pas mal, ca marche presque, sauf que le dernier caractère du fichier
  source est écrit deux fois dans le fichier destination. C'est normal. Voici
  ce que dit la page man correspondante:

  Remember fscanf returns EOF in case of errors or if it reaches eof. So you
  can check return value of fscanf instead of feof().
\begin{Verbatim}
while (fscanf(in_fd, "%s", username) != EOF) {
  fprintf(out_fd, "%s\n", username);
}  
\end{Verbatim}

\end{Reponse}

\Question Modifiez votre programme pour qu'il prenne ses arguments
(\texttt{source} et \texttt{destination}) depuis la ligne de commande grâce à
\texttt{argv}. Si aucun argument n'est fourni en ligne de commande, il faut
encore demander interactivement à l'utilisateur le nom des fichiers concernés.

\begin{Reponse}
  L'un des trucs faux très souvent rencontré est quelque chose comme
  \begin{Verbatim}
char input[32];
if (argc<2) {
  scanf("%s",input); 
} else {
  input=argv[1];
}
  \end{Verbatim}
Ca ne peut évidement pas marcher car on tente d'affecter un tableau dans un
autre, ce qui est interdit. Il faut soit changer input en un \texttt{char*}
(mais là, attention à bien réserver la place nécessaire pour l'éventuel scanf,
par exemple avec un malloc), soit utiliser \texttt{strcpy()} pour faire la
copie d'argv dans input.
\end{Reponse}


\Exercice\textbf{Les arguments de la ligne de commande sont des entiers.}

Lisez le code source \file{max2.c}, disponible dans le dépot. Il demande
interactivement deux entiers à l'utilisateur et renvoie la valeur maximum lue.

On souhaite modifier \file{max2} pour lui fournir les données en arguments de
la ligne de commande.

\Question Pour réaliser cela, un programmeur pressé a modifié \file{max2.c},
sous le nom \file{max2\_v2.c}. Examinez ce programme et compilez-le. Vous
obtenez un message d'erreur qui signifie que les parties gauches et droites
d'affectations ne sont pas du même type : en effet, {\tt a} et {\tt b} sont des
entiers, {\tt argv[1]} et {\tt argv[2]} sont des adresses (pointeurs). Exécutez
le code exécutable (qui est tout de même généré, car ces erreurs sont des {\em
  warnings}).

\Question Notre programmeur pressé a fait une première tentative de correction
en forçant une conversion de type, dans le fichier \file{max2\_v3.c}. Lisez-le,
compilez-le et vérifiez que l'erreur de compilation ne se produit plus.
Exécutez le programme obtenu ... et concluez.

\Question Corrigez le programme en utilisant la fonction \texttt{atoi}
(cf. \texttt{man atoi}).

\begin{Exercice}\textbf{Mesurer la complexité d'un fichier C.}

On souhaite avoir une mesure de la complexité de divers programmes C.
Pour cela, plusieurs métriques sont utilisables, que nous allons explorer au
fil des questions.

La première idée est de mesurer la longueur du texte. Il est assez courant
d'annoncer le nombre de lignes composant un programme donné (Noyau linux 2.6: 4
millions; GCC: 2,5 millions; Windows XP: 40 millions, ...).

\Question Faites un programme C \texttt{complexite} prenant le nom d'un fichier
en argument et comptant le nombre de lignes le composant (il faut compter les
occurences du caractère '$\backslash$n').

\Question Modifiez votre programme pour ne pas tenir compte des lignes blanches
(il faut utiliser un booléen réinitialisé à chaque ligne indiquant si on a vu
un caractère autre que '$\backslash$t' et '~').

\Question (facultative) Modifiez votre programme pour ne pas compter ce qui se
trouve à l'intérieur de commentaires. On souhaite traiter à la fois le style C
(\texttt{/* ... */}) et le style C++ (\texttt{// ... fin de ligne}). On
utilisera un booléen indiquant si on se trouve actuellement dans un commentaire
ou non.

\bigskip

Cette métrique de complexité est trompeuse car un long programme ne comportant
qu'un enchaînement de commandes sans \texttt{if}, \texttt{for} ou autres est
sans doute moins complexe qu'un programme plus court présentant un schéma
d'exécution plus complexe. 

\Question Modifiez votre programme pour qu'il compte les occurences du
caractère '\{', qui est présent dans la plupart des structures syntaxiques du
C. 

\Question Voici deux extensions possibles pour votre programme, à réaliser si
vous avez le temps:
\begin{itemize}
\item Comptez séparément les occurences des différentes constructions
  syntaxiques du C.% (\texttt{if}, \texttt{for}, \texttt{while}, \texttt{switch},
%  etc)
\item Faites en sorte que votre programme compte les lignes lorsqu'on lui passe
  l'argument supplémentaire \texttt{--long} tandis qu'il compte la complexité
  si on lui passe l'option \texttt{--complexe}. 
\end{itemize}
\end{Exercice}

\Exercice\textbf{Pour aller plus loin : les récipients en C.} 

\begin{Reponse}
  Pas de panique sur cet exo, il est optionnel. Le résoudre est sans doute plus
  complexe que le projet de TOP. C'est donc pour occuper ceux qui s'ennuient
  les soirs chez eux. Renvoyez ceux que ca intéresse vers moi, on gèrera par
  mail.
\end{Reponse}

Reprenez problème des récipients vu en TD et en TP dans le module de
TOP\footnote{Ces sujets sont disponibles sur la page
  \url{http://www.loria.fr/~quinson/teach-prog.html}}. Implémentez une solution
générique pour ce problème, sans limitation sur le nombre de récipients ni
borne arbitraire sur la profondeur d'exploration.  Utilisez l'algorithme le
plus efficace vu pour ce problème, c'est-à-dire celui stockant chaque état déjà
rencontrés sous forme d'un entier.

\begin{Reponse}
  Bon, le sujet de TOP est faux, et la façon de stocker dans un entier proposée
  ne marche pas. Faut que je corrige le problème de TOP, mais c'est une autre histoire.
\end{Reponse}

\Question Résolvez l'instance \{5,3\}$\rightarrow$4. C'est à dire que l'on dispose
de deux récipients de taille respective 5 et 3, et que l'on cherche à générer
la quantité 4. Comme présenté dans le sujet de TOP, cette instance a une
solution en 6 transvasements.

\Question Résolvez les instances suivantes 
\{8, 13, 21\}$\rightarrow$1 ; 
\{13, 21, 34\}$\rightarrow$1 ; 
\{21, 34, 55\}$\rightarrow$1 ; 
\{34, 55, 89\}$\rightarrow$1 ;
\{55, 89, 144\}$\rightarrow$1 ;
\{89, 144, 233\}$\rightarrow$1.
Pour les plus grosses de ces instances, un parcours en largueur de l'arbre des
possibilités est bien plus efficace que la méthode de backtracking proposée
dans le sujet de TOP.

\Question Résolvez l'instance \{1597, 2584, 4181\}$\rightarrow$1.
Pour cela, il vous faudra sans doute observer les solutions obtenues à la
question précédente, et «avoir une bonne idée»\ldots

\begin{Reponse}
  On observe en effet que quand on donne ainsi des nombres successifs de la
  suite de fibonacci, seuls les deux premiers récipients sont utilisés dans la
  solution optimale. Si on donne à l'algo de résolution les trois valeurs de
  cette question, le problème explose en mémoire et ca ne converge pas. Si on
  supprime le récipient 4181, une implémentation un peu efficace trouve en
  moins d'une seconde la solution, qui compte 5166 transvasements (quand même).
\end{Reponse}

\Question Peut-être existe-t-il un algorithme pour créer des instances de
problèmes arbitrairement complexe?

\begin{Reponse}
  Ce truc ressemble à une forme étendue du calcul du PGCD, je trouve. Y'a ptet
  moyen de trouver une idée. Je sais pas vraiment, une histoire avec beaucoup
  de taille de récipients premières entre elles, et la cible le pgcd d'un
  ensemble? 

  Le fait que les valeurs de fibbo donnent des instances aussi complexes me
  bluffe un peu. Y'a quelque chose que je comprend pas là dedans.
\end{Reponse}

% \medskip\noindent\textit{Si vous parvenez à résoudre cet exercice,
%   bravo. Faites un mail à \url{Martin.Quinson@loria,fr} pour montrer votre
%   solution (et partager vos instances pour augmenter le sujet l'an prochain ;)}

\end{document}

%%% Local Variables:
%%% coding: utf-8
