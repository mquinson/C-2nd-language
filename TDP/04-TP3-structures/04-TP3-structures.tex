\documentclass[10pt]{article}\usepackage[correction,nu]{esial}
\CSH\unA\sloppy
\usepackage[utf8]{inputenc}
\hypersetup{colorlinks=false,pdfborder={0 0 0}}

\title{TP3 : Structures en C}
\begin{document}

\maketitle

\Exercice \textbf{Gestion de points sur le plan.}

L'objectif de cet exercice est de définir un module de programme
gérant les points du plan en s'inspirant du design de code donné dans
le chapitre 4 du cours au sujet de la programmation modulaire en C.
 
\begin{Reponse}
Le but de cet exercice est d'introduire en douceur les structures et
le passage de paramètres par adresse aux fonctions.

J'ai choisi une structure simple dans laquelle on n'a que des nombres
contrairement à l'exercice ``annuaire'' qui utilise des chaînes de
caractères.
\end{Reponse}

\Question Définir une structure de données \texttt{point\_t} décrivant
un point et écrire une fonction constructeur ainsi qu'une fonction
destructeur. Vous ajouterez un champ \texttt{id} à votre structure,
contenant un entier unique identifiant chaque point, entier choisi par
le constructeur à la création de l'objet.

\Question Écrire une fonction qui permet d'afficher les différents
champs d'une structure \texttt{point\_t} avec le format {\em
  id(x,y)}. La fonction prend en argument un pointeur vers une
structure \texttt{point\_t}.

\Question Écrire un ensemble de tests dans un nouveau fichier (qui ne
chargera que le .h de votre module) et effectuant quelques actions sur
les points que vous définierez. Vous augmenterez cette classe de tests
au fur et à mesure de votre avancée dans les questions ci-dessous.

\Question Définir un \textit{copy constructor}, c'est-à-dire une
fonction prenant un pointeur vers une structure \texttt{point\_t} et
créant un nouvel objet étant la copie conforme du premier.

\Question Définissez et testez la fonction
\texttt{point\_move(point\_t *p, double dx, double dy)}.

\begin{Question} Écrire une fonction \framebox{\texttt{symetrique}}
  qui prend en argument un pointeur vers un \texttt{point\_t} ainsi
  qu'une option parmi {XX,YY,ORIG} selon laquelle, la fonction {\em
    symetrique} modifie les coordonnées d'un point avec celles de son
  symétrique par rapport à l'axe des x, des y ou par rapport à
  l'origine respectivement.

  On utilise une énumération pour déclarer les différentes options
  possibles. La fonction retourne 0 si l'opération se déroule
  correctement, et un code d'erreur sinon (1 par défaut, mais vous
  pouvez définir différents code de retour pour différents types
  d'erreurs).
\end{Question}

\Question Définissez une structure \texttt{polygone\_t} définie par un
ensemble de point constituant ses sommets. On pourra avoir au maximum
MAX sommets par polygone (avec MAX défini par un \#define dans le code).

\Question Écrivez un constructeur, un destructeur et une fonction
d'affichage pour \texttt{polygone\_t}.  Définissez et testez la
fonction \texttt{polygone\_translate(polygone\_t p, double dx, double
  dy)}.

\Question Calculez taille de la structure \texttt{point\_t} avec
l'opérateur \texttt{sizeof()}. Comparez cette taille à la somme des
tailles des différents éléments qui la composent. Que se passe-t-il si
le champ \texttt{id} est placé devant et devient de type
\texttt{short} ou \texttt{char}? 

\Question Afin de comprendre comment la structure est stockée en
mémoire, écrire une macro OFFSET(x,Y) qui donne pour un élément x de
la structure Y, son adresse relative par rapport à l'adresse de début
de la structure. Afficher l'offset de chaque élément dans les 2
structures. Conclure !

\begin{Reponse}
Le but de ces dernières questions et de les faire sensibiliser aux
problèmes d'alignements et les optimisations qu'un compilateur pourra
mettre en place pour optimiser le nombre d'accès mémoire ...

Aussi, c'est l'occasion de leur parler de la différence entre une
macro et une fonction.
\end{Reponse}

\bigskip\bigskip\Exercice
~
\begin{Reponse}
  Le carnet est à déclarer en global avant le main(). Sinon, il faut
  le passer en retour des fonctions, aussi, pour que la modification
  sur la copie soit repropagée à l'original.
\end{Reponse}

On souhaite créer un programme en C gérant un annuaire très simplifié qui
associe à un nom de personne un numéro de téléphone.

\Question Créer une structure \texttt{personne\_t} pouvant contenir
ces informations (nom et téléphone). Le nom peut contenir 32
caractères et le numéro 16 caractères. Écrivez le constructeur, le
destructeur et la fonction d'affichage de cette structure.

\Question Créer une nouvelle structure qui va représenter le carnet
d'adresses.  Cette structure \texttt{carnet\_t} contiendra un tableau
de 20 Personne et un compteur indiquant le nombre de personnes dans le
tableau. Écrivez le constructeur et le destructeur de cette structure.

\Question Créer une fonction qui ajoute une personne dans un carnet.

\Question Créer une fonction qui affiche un carnet.

\Question À partir des étapes précédentes, faire programme gérant un carnet
d'adresse. Créer un menu qui propose d'ajouter une nouvelle personne,
d'afficher le carnet ou de quitter.

\Question Quelques  extensions possibles:
\begin{itemize}
\item Ajout d'une fonction de sauvegarde de l'annuaire dans un
  fichier: Les données sauvegardées doivent être lues automatiquement
  au démarrage.
\item Taille dynamique de carnet: trouver une solution permettant
  d'avoir un nombre variable de personnes dans son carnet d'amis,
  potentiellement supérieur à 20.
\end{itemize}

% \bigskip\bigskip\Exercice

% Écrire un programme C qui permette de lire au clavier une suite de
% caractères représentant la plaque minéralogique (ancien format!) d'un véhicule
% (comme \texttt{982 BZZ 54}) et qui affiche à l'écran les plaques des 20
% véhicules immatriculés à la suite de celui-ci (dans le même département).

% Ainsi, si la chaîne lue est «{\tt 998 BZZ 54}», alors le programme affichera\\
% «{\tt 999 BZZ 54}», «{\tt 001 CAA 54}», «{\tt 002 CAA 54}», \dots,  
% «{\tt 019 CAA 54}».


% ;

\end{document}

%%% Local Variables:
%%% coding: utf-8
