\documentclass[10pt]{article}\usepackage[correction,nu]{esial}
\CSH\unA\sloppy
\usepackage[utf8]{inputenc}
\hypersetup{colorlinks=false,pdfborder={0 0 0}}

\title{TP3 : Structures en C}
\begin{document}

\maketitle

\begin{Reponse}
Le but de cet exercice est d'introduire en douceur les structures et
le passage de paramètres par adresse aux fonctions.

J'ai choisi une structure simple dans laquelle on n'a que des nombres
contrairement à l'exercice ``annuaire'' qui utilise des chaînes de
caractères.
\end{Reponse}

\begin{Question} Écrire une fonction \framebox{\texttt{symetrique}} qui prend
en argument :
\begin{itemize}
\item l'abscisse ($x \in \mathbb R$) d'un point,
\item son ordonnée ($y \in \mathbb R$),
\item et une option parmi {XX,YY,ORIG} selon laquelle, la fonction
  {\em symetrique} modifie les coordonnées d'un point avec celles de
  son symétrique par rapport à l'axe des x, des y ou par rapport à
  l'origine respectivement.
\end{itemize}
On utilise une énumération pour déclarer les différentes options
possibles. La fonction retourne 1 si l'opération se déroule
correctement, 0 sinon.
\end{Question}

\Question Dans la suite, un point est à définir par une structure {\em
  Point} avec les champs suivants :
\begin{itemize}
\item id : un nombre entier, identifiant du point,
\item x  : un nombre réel (float), l'abscisse du point,
\item y : un nombre réel (float), l'ordonnée du point.
\end{itemize}

\Question Écrire une fonction qui permet d'afficher les différents
champs d'une structure {\em Point} avec le format {\em id(x,y)}. La
fonction prend en argument une structure Point.

\Question Écrire une fonction qui permet d'initialiser un point et qui
prend comme argument une structure Point. 

\Question Récrire la fonction \framebox{\texttt{symetrique}} de telle
sorte qu'elle prenne en argument en plus de l'option, les coordonnées
du point sous forme d'une structure.

\Question Écrire une fonction qui permet de définir un segment de
droite comme un tableau de MAX points situés entre deux points
(extrémités) A et B. On utilisera \#define pour définir la valeur de
MAX.

\Question Pour afficher l'information sur l'ensemble des points
constituant le segment de droite, écrire une fonction
\framebox{\texttt{afficheSegment}} et qui fait appel à une fonction que
  vous avez déjà écrite.

\Question On s'intéresse maintenant à la taille de la structure
Point. Afficher la taille de la structure et vérifier bien que sa
taille correspond à la somme des tailles des différents éléments qui la
composent.

\Question On définit une structure Point2 similaire à Point mais qui a
son identifiant codé comme étant {\bf short}. Même question pour cette
nouvelle structure. Que constate -t- on ?

\Question Afin de comprendre comment la structure est stockée en
mémoire, écrire une macro OFFSET(x,Y) qui donne pour un élément x de
la structure Y, son adresse relative par rapport à l'adresse de début
de la structure. Afficher l'offset de chaque élément dans les 2
structures. Conclure !

\begin{Reponse}
Le but de ces dernières questions et de les faire sensibiliser aux
problèmes d'alignements et les optimisations qu'un compilateur pourra
mettre en place pour optimiser le nombre d'accès mémoire ...

Aussi, c'est l'occasion de leur parler de la différence entre une
macro et une fonction.
\end{Reponse}

\newpage

\bigskip\bigskip\Exercice
~
\begin{Reponse}
  Le carnet est à déclarer en global avant le main(). Sinon, il faut
  le passer en retour des fonctions, aussi, pour que la modification
  sur la copie soit repropagée à l'original.
\end{Reponse}

On souhaite créer un programme en C gérant un annuaire très simplifié qui
associe à un nom de personne un numéro de téléphone.

\Question Créer une structure Personne pouvant contenir ces informations (nom
et téléphone). Le nom peut contenir 32 caractères et le numéro 16
caractères.

\Question Créer une nouvelle structure qui va représenter le carnet d'adresses.
Cette structure Carnet contiendra un tableau de 20 Personne et un compteur
indiquant le nombre de personnes dans le tableau.

\Question Créer ensuite une fonction qui renvoie une structure Personne en
prenant en argument un nom et un téléphone.

\Question Rajouter une fonction qui affiche les informations contenues dans la
structure Personne passée en argument.

\Question Créer une fonction qui ajoute une personne dans un carnet.

\Question Créer une fonction qui affiche un carnet.

\Question À partir des étapes précédentes, faire programme gérant un carnet
d'adresse. Créer un menu qui propose d'ajouter une nouvelle personne,
d'afficher le carnet ou de quitter.

\medskip
Une extension possible serait d'ajouter une fonction de sauvegarde de
l'annuaire dans un fichier, et de faire en sorte que les données sauvegardées
puissent être lues automatiquement au démarrage. 

\bigskip\bigskip\Exercice

Écrire un programme C qui permette de lire au clavier une suite de
caractères représentant la plaque minéralogique (ancien format!) d'un véhicule
(comme \texttt{982 BZZ 54}) et qui affiche à l'écran les plaques des 20
véhicules immatriculés à la suite de celui-ci (dans le même département).

Ainsi, si la chaîne lue est «{\tt 998 BZZ 54}», alors le programme affichera\\
«{\tt 999 BZZ 54}», «{\tt 001 CAA 54}», «{\tt 002 CAA 54}», \dots,  
«{\tt 019 CAA 54}».




\end{document}

%%% Local Variables:
%%% coding: utf-8
