\documentclass[10pt]{article}\usepackage[nu]{esial}
\CSH\sloppy
\usepackage{url,moreverb}
\usepackage[utf8]{inputenc}
\hypersetup{colorlinks=false,pdfborder={0 0 0}}

\sloppy
\begin{document}
\title{TP5: Allocation mémoire et mise au point de programmes C}
\maketitle

\newcommand{\I}{\hspace{1.5em}}


L'objectif de ce TP est de vous donner des pistes sur comment trouver les bugs
dans vos programmes.

La recherche d'une erreur au sein d'un programme est une sorte de jeu de pistes
où l'on recherche des informations sur le contexte, les symptômes, les causes
possibles de l'erreur. Cela permet de déterminer sa localisation et la manière
de la corriger.  La méthode traditionnelle consistent à utiliser la commande
\texttt{printf} en divers endroits du programme est l'expression de cette
recherche d'information. Des outils tels que \cmd{gdb} et \cmd{valgrind}
facilitent l'obtention d'informations sur les programmes.


\begin{Exercice}\textbf{la méthode printf}

  \noindent\begin{minipage}{.76\linewidth}
    Cette méthode est utilisée dans les cas où on ne peut (ou ne veut) pas
    utiliser de debugger. Attention cependant au piège classique de cette
    méthode, mis en valeur dans le programme \texttt{boom.c} ci-contre
    (également dans le dépot). 

    Ce programme devrait afficher \framebox{\texttt{12Erreur de segmentation}}
    puisque la ligne 9 revient à déréférencer le pointeur NULL, ce qui est
    interdit.
    
    \Question Quel est l'affichage généré par ce programme ?

    C'est parce que les affichages de printf ne sont pas toujours réalisées
    immédiatement. Pour des raisons de performances, le système cherche en
    effet à retarder les affichages de façon à avoir moins d'action d'affichage
    pour plus de texte à chaque fois. C'est pourquoi les ``1'' et ``2'' sont
    placés dans un tampon pour être affichés plus tard. Malheureusement, comme
    l'erreur de segmentation de la ligne 9 tue brutalement le programme, ces
    messages ne seront jamais affichés.



  \end{minipage}~ \begin{minipage}{.23\linewidth}
    \VerbatimInput[label=boom.c,numbers=right]{sources/boom.c}
  \end{minipage}

  \medskip
  Les \texttt{printf} suggèrent donc une localisation erronée du problème, ce
  qui peut faire perdre un temps considérable. Plusieurs solutions permettent
  d'éviter ou au moins de contrôler cette mise en tampon.

  \Question Ajoutez des retours-chariots à la fin des affichages (la ligne 6
  devient \texttt{printf("1\underline{$\backslash$n}");}). Quel est maintenant
  l'affichage de votre programme ? Et si vous lancez votre programme de la
  façon suivante : \cmd{./boom|less} ?

  C'est parce que le système vide le tampon à chaque fin de ligne si et
  seulement l'affichage est dirigé sur un terminal.

  \Question Retirez les $\backslash$n que vous aviez ajouté à la question
  précédente, et demandez à réaliser les affichages sur la sortie d'erreur (en
  utilisant \cmd{fprintf(stderr, "...")} à la place de printf. Quel est
  maintenant le comportement de votre programme? Et si la sortie n'est pas un
  terminal mais un tube?

  C'est parce que la sortie d'erreur n'est pas mise en tampon, car les messages
  d'erreurs sont considérés urgents et doivent être affichés au plus vite, même
  si cela induit une perte de performances.

  \Question Rechangez vos affichages pour utiliser la sortie standard (avec
  \cmd{printf}), et ajoutez des \cmd{fflush(stdout)} après chaque \cmd{printf}.
  Quel est maintenant le comportement de votre programme? Et si la sortie n'est
  pas un terminal mais un tube?

  C'est parce que la fonction \cmd{fflush} a pour objectif de pour vider le
  tampon et forcer l'affichage immédiat des informations.

  \medskip

  \noindent\textbf{Conclusion.} Cet exercice nous a permis d'explorer le
  principal piège de la mise au point à base de \cmd{printf}. Nous avons vu 3
  façons de contourner ce piège, mais cette méthode reste artisanale, et il est
  souvent nécessaire d'utiliser des outils spécialisés comme \cmd{gdb}.
\end{Exercice}
\newpage

\begin{Exercice}\textbf{le debugger GNU: gdb (utilisation de base)}

  \noindent\begin{minipage}[b]{.6\linewidth}
    Nous utiliserons comme premier exemple le programme \texttt{boucle.c}
    ci-contre (également dans le dépot).  \smallskip

    \textbf{Pour le compiler}, il convient d'utiliser la commande \run{gcc -Wall
      -Wextra -g -o boucle boucle.c}. L'option \texttt{-Wall} demande
    l'activation de nombreux \textit{warnings}, mais pas tous puisque
    \texttt{-Wextra} en active d'autres (il en manque encore).  \texttt{-g}
    ajoute au binaire produit les informations de deboguage nécessaire à
    \cmd{gdb} (et autres debuggers).

    \Question Exécutez ce programme. Que constatez vous?
    \smallskip

    \textbf{Lancement de gdb}. Tapez la commande \run{gdb ./boucle} pour
    charger votre programme dans l'environnement GDB.  On controle ce programme
    en tapant des commandes à l'invite. Les commandes les plus importantes sont
    \texttt{help}, \texttt{list}, \texttt{quit} et \texttt{run}.

    \Question Essayez la session suivante dans gdb:
    \begin{itemize}
    \item[$\bullet$] Chargez boucle dans gdb et lancez le programme. 
    \item[$\bullet$] Tapez $<$ctrl+c$>$ pour interrompre votre programme.
    \item[$\bullet$] Visualisez le code en cours d'exécution avec
      \texttt{list}.\vspace{.2\baselineskip}
    \end{itemize}

  \end{minipage}\hfill\begin{minipage}[b]{.38\linewidth}
    \VerbatimInput[label=boucle.c,numbers=right]{sources/boucle.c}
  \end{minipage}

  \vspace{.2\baselineskip}\noindent
  \begin{minipage}{\linewidth}
  \begin{itemize}
  \item[$\bullet$] Reprenez l'exécution avec \texttt{cont}, puis interrompez-la
    de nouveau. L'exécution n'a pas progressé.
  \item[$\bullet$] Aidez le programme à franchir la zone difficile à l'aide de
    la commande \cmd{jump 11}, ce qui fait sauter l'exécution à la ligne 11
    (oui, cela modifie l'exécution du programme). Le programme doit
    se terminer normalement. Reste à comprendre pourquoi le programme ne passe
    pas la ligne 10 seul.
  \end{itemize}
  \end{minipage}

  \smallskip
  \noindent \textbf{Points d'arrêt et exécution pas à pas}

  Lors de la traque d'une erreur, il est fréquent d'avoir une idée de sa
  localisation potentielle. \cmd{gdb} permet donc de spécifier des points
  d'arrêt dans le code où l'exécution est automatiquement interrompue.  La
  commande \cmd{break} suivie d'un nom de fonction ou d'un numéro de ligne
  (éventuellement associé à un fichier) insère un point d'arrêt à l'endroit
  spécifié. \cmd{clear} supprime le point d'arrêt spécifié.

  Placez un point d'arrêt sur la fonction \texttt{main} puis lancez
  l'exécution. Elle s'interrompt avant le début du code.  Expérimentez avec les
  commandes \cmd{next} et \cmd{step}. Chacune permet d'avancer l'exécution
  d'une ligne puis de bloquer l'exécution.  Si cette ligne contient un appel de
  fonction, \cmd{step} entre dans le code de cette fonction tandis que
  \cmd{next} l'exécute en entier et passe à la ligne suivante de la fonction
  courante.

  \Question Pour trouver le problème, interrompez au besoin votre programme
  (ctrl-C), utilisez la commande \cmd{print} pour afficher le contenu de la
  variable i (\cmd{print i}). Vous pouvez également le faire continuer
  (commande \cmd{continue}), et le réinterrompre. Corrigez le problème.

  \textit{Indice:} ce premier bug se trouve ligne 8.

  \Question Maintenant que le programme s'exécute jusqu'à la fin, on constate
  que l'affichage d'une des cases de \cmd{tab} après un arrêt à la ligne 21
  indique que l'affectation du tableau ne s'effectue pas correctement,
  puisqu'elles valent 0 au lieu du 1 attendu. Réexécutez votre programme pas à
  pas pour comprendre le problème, puis corrigez le.

  \textit{Indice:} ce second bug se trouve ligne 10.
\end{Exercice}

\begin{Exercice}\textbf{le debugger GNU: gdb (utilisation avec les fonctions)}

  Nous allons maintenant utiliser le debugger avec un autre programme afin
  d'expérimenter les opérations permettant de trouver les problèmes impliquant
  des fonctions.

  \textbf{Pile et cadres} La commande \cmd{backtrace} permet d'afficher la pile
  d'exécution du processus. Compilez \file{fact.c} (page suivante et dans le
  dépot) puis chargez fact dans \cmd{gdb}. Spécifiez un point d'arrêt sur la
  ligne 9 (x=1) et lancez l'exécution. Lorsque le processus est stoppé,
  exécutez \cmd{backtrace}.

  La liste affichée indique tout d'abord les appels récursifs à fact et
  termine par main. Les fonctions sont donc listées depuis l'appel le plus
  imbriqué (regardez la valeur indiquée pour le paramètre n de f pour chaque
  cadre) vers l'appel le moins imbriqué (donc dans l'ordre inverse de l'ordre
  chronologique, d'où le nom de la commande).

  Chaque ligne constitue ce que l'on appelle un \textit{cadre de pile} («frame»
  en étranger). Il est possible de se déplacer dans la pile avec les commandes
  \cmd{up} et \cmd{down}, ou directement avec la commande \cmd{frame} suivie du
  numéro de cadre visé.

  \noindent\begin{minipage}{.67\linewidth}
    \medskip \textbf{Affichage de variables et d'expressions} La commande
    \cmd{print} permet d'afficher le contenu d'une variable. Placez un point
    d'arrêt sur \texttt{fact} puis ré-exécutez. Utilisez \run{print n}. La
    commande \cmd{disp} est similaire, mais affiche le résultat à chaque
    interruption du programme. Exécutez \run{disp (char)n+65} puis utilisez
    \cmd{cont} plusieurs fois.
   
    \I On peut de plus modifier des valeurs avec \run{set variable VAR=EXP} où
    VAR est le nom de la variable à modifier et EXP l'expression dont le
    résultat est à lui affecter. Si le nom de la variable à modifier n'entre
    pas en conflit avec les variables internes de GDB, on peut omettre le
    mot-clé variable.

  \medskip
  \noindent\textbf{Conclusion sur gdb}. Vous en savez maintenant assez
  sur \cmd{gdb} pour faire vos premiers pas. Il existe cependant de nombreuses
  fonctionnalités que nous n'avons pas abordé ici comme les \textit{watchpoints}
  (qui arrêtent l'exécution quand une variable donnée est modifiée), le
  chargement de fichiers \texttt{core} pour débugger un programe après sa mort,
  la prise de contrôle de processus en cours d'exécution, et bien d'autres
  encore. \run{info gdb} pour les détails.
  \end{minipage}\hfill\begin{minipage}{.32\linewidth}
    \VerbatimInput[label=fact.c,numbers=right]{sources/fact.c}
  \end{minipage}


\end{Exercice}

\begin{Exercice}\textbf{La suite d'outils valgrind}
  
  \cmd{valgrind} est une suite d'outil fabuleuse pour mettre au point vos
  programmes. Selon l'outil utilisé, il est possible de détecter la plupart des
  problèmes liés à la mémoire (outil \texttt{memcheck}), d'étudier les effets
  de cache pour améliorer les performances (avec \texttt{cachegrind}), de
  débugger des programmes multi-threadés (avec \texttt{hellgrind}, voir le
  cours de système en 2A) ou encore d'optimiser les programmes (avec
  \texttt{callgrind}). Nous allons nous intéresser au premier outil, que l'on
  lance avec \run{valgrind --tool=memcheck ./prog}

  \Question Lancez valgrind sur le programme \texttt{boom} étudié plus
  tôt. S'affichent de nombreuses lignes commençant par \texttt{==$<$identifiant
    du processus$>$==}. Elles sont le fait de \texttt{valgrind}.

  La cause de l'erreur de segmentation est donnée par le second groupe de
  ligne:
  \begin{Verbatim}
==29579== Invalid write of size 4
==29579==    at 0x80483CA: main (boom.c:9)
==29579==  Address 0x0 is not stack'd, malloc'd or (recently) free'd    
  \end{Verbatim}

  À la ligne \texttt{boom.c$:$9}, nous écrivons 4 octets (sans doute un entier)
  à un endroit invalide. En effet, l'adresse \texttt{0x0} [où nous tentons
  d'écrire] n'est ni sur la pile, ni le résultat d'un \texttt{malloc} et il n'a
  pas été free()é récemment. Bien sûr! La ligne 9 écrit à l'adresse pointé par
  \texttt{p}, mais \texttt{p} vaut la valeur \texttt{NULL}, qui n'est pas une
  adresse valide (et on a \texttt{NULL=0x0}). \texttt{valgrind} localise
  immédiatement et précisément le problème.

  \Question Lancez maintenant \cmd{valgrind} sur le programme \texttt{boucle}
  (après avoir corrigé les deux bugs identifiés dans l'exercice 2). Vous pouvez
  constater que le programme que l'on croyait corrigé contient encore des
  problèmes:
  \begin{Verbatim}
==10816== Invalid write of size 4
==10816==    at 0x8048429: initialise (boucle.c:12)
==10816==    by 0x8048476: main (boucle.c:20)
==10816==  Address 0x41a7c68 is 0 bytes after a block of size 40,000 alloc'd
==10816==    at 0x402601E: malloc (vg_replace_malloc.c:207)
==10816==    by 0x8048465: main (boucle.c:19)    
  \end{Verbatim}

  La ligne \texttt{boucle.c$:$12} tente d'écrire 4 octets à un endroit
  invalide. De plus, cet endroit est localisé juste après un gros bloc mémoire
  alloué en \texttt{boucle.c$:$19}. Corrigez ce problème (indice: le bug est en
  ligne 10).

  \Question Relancez \cmd{valgrind} sur le programme \texttt{boucle}.  À la fin
  de l'exécution, valgrind affiche:
  \begin{Verbatim}
==21800== LEAK SUMMARY:
==21800==    definitely lost: 0 bytes in 0 blocks
==21800==    indirectly lost: 0 bytes in 0 blocks
==21800==      possibly lost: 0 bytes in 0 blocks
==21800==    still reachable: 40,000 bytes in 1 blocks
==21800==         suppressed: 0 bytes in 0 blocks
==21800== Rerun with --leak-check=full to see details of leaked memory
  \end{Verbatim}
  
  Il y a donc un bloc de mémoire (de 40 ko) obtenu par \texttt{malloc},
  mais jamais restitué au système avec \texttt{free}. Ajoutez les options
  nécessaires pour voir lequel et corrigez le problème.
\end{Exercice}
\end{document}

%%% Local Variables:
%%% coding: utf-8
